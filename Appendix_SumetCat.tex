
\section{Control plots}


\begin{figure}[h]
\centering
\subfloat[]{\includegraphics[trim={0, 4cm, 0, 4cm},clip,width=.4\textwidth]{figure/ZpT_MJPlots/Zmumu_SR_Logy_isolation_5TeV.pdf}}
\subfloat[]{\includegraphics[trim={0, 4cm, 0, 4cm},clip,width=.4\textwidth]{figure/ZpT_MJPlots/Zmumu_CR_Logy_isolation_5TeV.pdf}}

\subfloat[]{\includegraphics[trim={0, 4cm, 0, 4cm},clip,width=.4\textwidth]{figure/ZpT_MJPlots/Zmumu_SR_Logy_isolation_13TeV.pdf}}
\subfloat[]{\includegraphics[trim={0, 4cm, 0, 4cm},clip,width=.4\textwidth]{figure/ZpT_MJPlots/Zmumu_CR_Logy_isolation_13TeV.pdf}}

\caption{Isolation variable $p_{T}^{cone \Delta R=0.2}/p_{T}^{\mu}$ for the sub-leading muon candidate in Data (black) and MC (blue) at 5~TeV (top) and 13~TeV (bottom). The estimated number of MJ events in the full phase space of the SR is calculated by subtracting the MC events from the Data events in each region to find $N_{MJ}$, and then solving for $N_{MJ}^{SR} = N_{MJ}^{B} \times \frac{N_{MJ}^{C}}{N_{MJ}^{D}}$.}
\label{f:Iso_Zmm}
\end{figure}