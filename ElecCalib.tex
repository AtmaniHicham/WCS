
Reference~\cite{Xu:2657152} details the electrons performance studies done using the special low-pile-up datasets collected by the ATLAS detector at
$\sqrt{s} = 5$ and 13~\TeV.
The electron reconstruction scale factors are obtained by extrapolation of the standard high-pileup SFs to the \lowmu regime, and apply to both the 13 and 5 \TeV{} datasets. The identification scale factors are measured in-situ separately for the 13 and 5 \TeV{} data.
Isolation and trigger efficiencies and SF are measured in-situ for the 13 and 5 \TeV{} data sets.
For the electron energy calibration the global strategy follows the methodology applied in high-$\mu$ standard calibration and described in Ref.~\cite{Aad:2019tso}.
Electron energy scale and resolution corrections are measured in-situ using \Zee events from the low pile-up dataset. 
%%The main limitation of the insitu performance studies is the limited statistic of the  \Zee samples in each of the 5 and 13~\TeV datasets.
