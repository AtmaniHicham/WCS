\subsection{Results}
\label{sec:Zresult}

Here we present the unfolded distributions for the \Zboson analysis. We begin with the unfolded differential cross-sections for both the dilepton transverse momentum $\left(p_{T}^{ll} \right)$ and the hadronic recoil $\left(u_{T}\right)$ along with their respective uncertainties in each \Zboson decay channel: \Zee and \Zmm. Our final result compares the normalized differential cross-section to various Monte Carlo generators. All normalized cross-sections are normalized by the integrated cross-section in the fiducial phase space. The results are shown separately for the 13 \TeV ~(\ref{ssec:Zunf13TeV}) and 5 \TeV ~(\ref{ssec:Zunf5TeV}) datasets. All \ptdilep results are unfolded with 2 iterations. The \ut results are unfolded with 5 iterations at 5 TeV with `5GeV` binning and 10 iterations at 13 TeV with '7GeV' binning. %Figure~\ref{fig:biasunc_min_13} shows that the total uncertainty is minimized when \ut is unfolded with approximately 5 iterations at 13 TeV.
Two compatibility checks are performed in sections~\ref{sssec:compatibilityHighmu}~and~\ref{ssec:pTZcompatibility}.



%%%%%%%%%%%%%%%%%%%%%%%%%%%%%%%%%%%%%
%%%%%%%%%%%%%%%%%%%%%%%%%%%%%%%%%%%%%
\subsubsection{$Z$ unfolded results at $\sqrt{s} = 13$~\TeV}
\label{ssec:Zunf13TeV}
%%%%%%%%%%%%%%%%%%%%%%%%%%%%%%%%%%%%%
%%%%%%%%%%%%%%%%%%%%%%%%%%%%%%%%%%%%%

Figures~\ref{f:unf_pT_Zee13} and~\ref{f:unf_pT_Zmm13} show the differential cross section distributions of the dilepton transverse momentum for the \Zee and \Zmm channels, respectively. The nominal Monte Carlo generator (PowhegPythia8) overestimates the distribution at low $p_{T}$ i.e. $p_{T} < 10 \textrm{ GeV}$. This effect is consistent with what is seen at reconstructed-level (Figs.~\ref{f:ZpT_Zee13} and~\ref{f:ZpT_Zmm13}). Both channels then agree (within uncertainty) with the MC distribution until about 40 \GeV, when the MC begins to underestimate the unfolded distribution. This effect is also seen at reconstructed-level. The \Zmm channel (\ref{f:unf_pT_Zmm13}) shows a positive statistical fluctuation at 20 \GeV that is also seen at reconstructed-level (\ref{f:ZpT_Zmm13}).

\begin{figure}[h]
\centering
\subfloat[]{\includegraphics[width=.49\textwidth]{figure/Z_Plots_unfolding_results/unfold_v20210511/XSec_v20210511_13TeV_Zee_2GeVBin_2iters.pdf}\label{f:unf_pT_Zee13}}
\subfloat[]{\includegraphics[width=.49\textwidth]{figure/Z_Plots_unfolding_results/unfold_v20210511/XSec_v20210511_13TeV_Zmumu_2GeVBin_2iters.pdf}\label{f:unf_pT_Zmm13}}

\subfloat[]{\includegraphics[width=.49\textwidth]{figure/Z_Plots_unfolding_results/unfold_v20210511/XSec_v20210511_13TeV_Zee_finerT_Rebin7_uT_10iters.pdf}\label{f:MCData_Zee_uT13_rebin7}}
\subfloat[]{\includegraphics[width=.49\textwidth]{figure/Z_Plots_unfolding_results/unfold_v20210511/XSec_v20210511_13TeV_Zmumu_finerT_Rebin7_uT_10iters.pdf}\label{f:MCData_Zmm_uT13_rebin7}}

\caption{Unfolded dilepton transverse momentum differential cross section distribution in the electron (left) and muon (right) \Zboson decay channels for the $\sqrt{s} = 13$~\TeV\ dataset. The cross section is shown in the upper panel and the lower panel shows the ratio of unfolded data to the nominal (reweighted) Powheg+Pythia8 MC. The uncertainties in the ratio plot are split to show the statistical uncertainty (black error bars), and experimental systematic uncertainies including luminosity (red band). The total uncertainty is the same total uncertainty as that shown in figures \ref{f:AlluncZee_pT13TeV} to \ref{f:AlluncZmm_uT13TeV_7GeV}.}\end{figure}


Figures~\ref{f:MCData_Zee_uT13_rebin7} and ~\ref{f:MCData_Zmm_uT13_rebin7} show the differential cross section distributions of the hadronic recoil for the \Zee and \Zmm channels, respectively. Comparing the two unfolded observables $p_{T}^{ll}$ and $u_{T}$ and referring back to their corresponding correlation matrices (figs.~\ref{f:corrMatrixZee_pT13TeV} -~\ref{f:corrMatrixZmm_uT13TeV}), one can see the effect of the strongly correlated bins in the $u_{T}$ observable, leading to the sinusoidal pattern in its Bayesian-unfolded result.

\begin{figure}[h]
\centering
\subfloat[]{\includegraphics[width=.49\textwidth]{figure/Z_Plots_unfolding_results/ComDataPred/XSec_Comp_13TeV_ee_2GeVBin.pdf}\label{f:MCData_Zee_pT13}}
\subfloat[]{\includegraphics[width=.49\textwidth]{figure/Z_Plots_unfolding_results/ComDataPred/XSec_Comp_13TeV_mumu_2GeVBin.pdf}\label{f:MCData_Zmm_pT13}}

\subfloat[]{\includegraphics[width=.49\textwidth]{figure/Z_Plots_unfolding_results/ComDataPred/XSec_Comp_13TeV_ee_finerT_Rebin7_uT.pdf}\label{f:MCData_Zee_uT13}}
\subfloat[]{\includegraphics[width=.49\textwidth]{figure/Z_Plots_unfolding_results/ComDataPred/XSec_Comp_13TeV_mumu_finerT_Rebin7_uT.pdf}\label{f:MCData_Zmm_uT13}}
\caption{Comparison between various Monte Carlo generator predictions and the unfolded dilepton transverse momentum differential cross-section distribution in the electron (left) and muon (right) \Zboson decay channels for the $\sqrt{s} = 13$~\TeV\ dataset. The differential cross-section has been normalized by the total cross-section. The unfolded data and MC distributions are shown in the upper panel and the lower panel shows the ratio of the various distributions to the data. In the lower panel, the total systematic uncertainties are shown in light grey, and the total overall uncertainty (stat. plus sys.) is shown in dark grey.}\label{f:MCData_Z_13}
\end{figure}


Figures ~\ref{f:MCData_Zee_pT13}, ~\ref{f:MCData_Zmm_pT13} \& ~\ref{f:MCData_Zee_uT13}, ~\ref{f:MCData_Zmm_uT13} compare various Monte Carlo generator predictions for the differential \pTZ cross-section to the data as a function of \pT. The differential cross-sections are normalized by the total cross-section. We use the MCs Sherpa (v2.2.1), DYTurbo (to NNLL+NNLO), Powheg+Pythia8 AZNLO Tune (the nominal MC) and Pythia8 with AZ tune. At low \pT, Pythia8 with AZ tuning is most consistent with the data, while at high \pT the data is best represented by Sherpa.

\clearpage



%%%%%%%%%%%%%%%%%%%%%%%%%%%%%%%%%%%%%
%%%%%%%%%%%%%%%%%%%%%%%%%%%%%%%%%%%%%
\subsubsection{$Z$ unfolded results at $\sqrt{s} = 5$~\TeV}
\label{ssec:Zunf5TeV}
%%%%%%%%%%%%%%%%%%%%%%%%%%%%%%%%%%%%%
%%%%%%%%%%%%%%%%%%%%%%%%%%%%%%%%%%%%%

\begin{figure}[h]
\centering
\subfloat[]{\includegraphics[width=.49\textwidth]{figure/Z_Plots_unfolding_results/unfold_v20210511/XSec_v20210511_5TeV_Zee_2GeVBin_2iters.pdf}\label{f:unf_pT_Zee5}}
\subfloat[]{\includegraphics[width=.49\textwidth]{figure/Z_Plots_unfolding_results/unfold_v20210511/XSec_v20210511_5TeV_Zmumu_2GeVBin_2iters.pdf}
\label{f:unf_pT_Zmm5}}

\subfloat[]{\includegraphics[width=.49\textwidth]{figure/Z_Plots_unfolding_results/unfold_v20210511/XSec_v20210511_5TeV_Zee_5GeVBin_uT_5iters.pdf}\label{f:unf_uT_Zee5}}
\subfloat[]{\includegraphics[width=.49\textwidth]{figure/Z_Plots_unfolding_results/unfold_v20210511/XSec_v20210511_5TeV_Zmumu_5GeVBin_uT_5iters.pdf}\label{f:unf_uT_Zmm5}}
\caption{Unfolded dilepton transverse momentum differential cross-section distribution in the electron (left) and muon (right) \Zboson decay channels for the $\sqrt{s} = 5$~\TeV\ dataset. The cross section is shown in the upper panel and the lower panel shows the ratio of unfolded data to the nominal (non-reweighted) Powheg+Pythia8 MC. The uncertainties in the ratio plot are split to show the statistical uncertainty (black error bars), and experimental systematic uncertainies including luminosity (red band). The total uncertainty is the same total uncertainty as that shown in figures \ref{f:AlluncZee_pT5TeV} to \ref{f:AlluncZmm_uT5TeV}.}\end{figure}

Figures~\ref{f:unf_pT_Zee5} and~\ref{f:unf_pT_Zmm5} show the differential cross-section distributions of the dilepton transverse momentum for the \Zee and \Zmm channels, respectively, while figures~\ref{f:unf_uT_Zee5} and~\ref{f:unf_uT_Zmm5} show the differential cross-section distributions of the hadronic recoil for the \Zee and \Zmm channels.
The unfolded results are consistent with the nominal MC aside from a statistical fluctuation at 30 \GeV\ that is also seen at reconstructed-level (Fig.~\ref{f:ZpT_Zee5} and~\ref{f:ZpT_Zmm5}). Due to its poorer resolution, the hadronic recoil result is once again highly regularized (strong bin-to-bin correlations as is shown in figs.~\ref{f:corrMatrixZee_uT5TeV} and~\ref{f:corrMatrixZmm_uT5TeV}) and so the unfolded data is not able to capture the shape as well as in the \ptll\ measurement.


\begin{figure}[h]
\centering
\subfloat[]{\includegraphics[width=.49\textwidth]{figure/Z_Plots_unfolding_results/ComDataPred/XSec_Comp_5TeV_ee_2GeVBin.pdf}\label{f:MCData_Zee_pT5}}
\subfloat[]{\includegraphics[width=.49\textwidth]{figure/Z_Plots_unfolding_results/ComDataPred/XSec_Comp_5TeV_mumu_2GeVBin.pdf}
\label{f:MCData_Zmm_pT5}}

\subfloat[]{\includegraphics[width=.49\textwidth]{figure/Z_Plots_unfolding_results/ComDataPred/XSec_Comp_5TeV_ee_5GeVBin_uT.pdf}\label{f:MCData_Zee_uT5}}
\subfloat[]{\includegraphics[width=.49\textwidth]{figure/Z_Plots_unfolding_results/ComDataPred/XSec_Comp_5TeV_mumu_5GeVBin_uT.pdf}
\label{f:MCData_Zmm_uT5}}
\caption{Comparison between various Monte Carlo generator predictions and the unfolded dilepton transverse momentum differential cross-section distribution in the electron (left) and muon (right) \Zboson decay channels for the $\sqrt{s} = 5$~\TeV\ dataset. The differential cross-section has been normalized by the total cross-section. The unfolded data and MC distributions are shown in the upper panel and the lower panel shows the ratio of the various distributions to the data. In the lower panel, the total systematic uncertainties are shown in light grey, and the total overall uncertainty (stat. plus sys.) is shown in dark grey.}\label{f:MCData_Z_5}
\end{figure}

Figures~\ref{f:MCData_Zee_pT5},~\ref{f:MCData_Zmm_pT5} \& \ref{f:MCData_Zee_uT5},~\ref{f:MCData_Zmm_uT5} compare various Monte Carlo generator predictions for the differential \pTZ cross-section to the data as a function of \pT in the 5 \TeV\ decay channels. The differential cross-sections are normalized by the total cross-section. The predictions at 5 \TeV\ are more consistent with the data than at 13 \TeV. At low \pT, both tunings of Pythia perform well. We see a difference in the generator predictions at high \pT between decay channels for both observables: in the \Zee channel DYTurbo and the Pythia8 AZ tuning perform well, whereas in the \Zmm channel only Sherpa performs well.


\clearpage
%%%%%%%%%%%%%%%%%%%%%%%%%%%%%%%%%%%%%%%
%%%%%%%%%%% COMPATIBILITY  %%%%%%%%%%%%%%%%%%%%%%%%%%%%%%%%%%%%%%%%%%%%%%%%
\subsubsection{Compatibility study of the low-$\mu$ $p_{T}^{Z}$ measurements between the low-$\mu$ and the high-$\mu$ analysis}
\label{sssec:compatibilityHighmu}

Here we present the comparison of the low pile-up $p_{\mathrm{T}}^{Z}$ measurement at 13~TeV (\ref{sec:physcorr}) to the recently published ATLAS measurement of $p_{\mathrm{T}}^{Z}$ at 13~TeV using the full high pile-up dataset Ref~\cite{Aad:2019wmn}.
The high-$\mu$ $p_{\mathrm{T}}^{Z}$ distribution from~Ref~\cite{Aad:2019wmn}  is shown in \Fig{\ref{f:highmuZpT}}.
The electron and muon channels are compared separately. Since the fiducial phase spaces in the two analyses are different~(\Tab{\ref{tab:fiducial}}), corresponding corrections must be taken into account.

\begin{figure}[h]
\centering
\includegraphics[width=.79\textwidth]{figure/ZpT_highmu_compatibility/Plot_TheoryComparison_PTZ.pdf}
\caption{Unfolded p$_{T}^{Z}$ results in the high-$\mu$ analysis.}
\label{f:highmuZpT}
\end{figure}

The only difference in fiducial phase space is the lepton $p_\mathrm{T}$ cut, which is 25~GeV in the low-$\mu$ analysis and 27~GeV in the high-$\mu$ analysis because in high-$\mu$ data, the trigger for $p_\mathrm{T}^l$ is 27GeV.
A corresponding correction factor is applied to the high-$\mu$ distribution before comparison~(\Fig{\ref{f:Correction}}). This correction factor is derived from the ratio of events in low-mu fiducial phase space to high-mu fiducial phase space, obtained from MC estimation.

\begin{table}[h]
\centering
\begin{tabular}{|c|c|c|}
\hline
Fiducial & Low-$\mu$ Analysis & High-$\mu$ Analysis \\
\hline
$\eta_l$  & $|\eta_{l}| < 2.5$ & $|\eta_{l}| < 2.5$ \\
\hline
$p_{T}^{l}$  & $p_{T}^{l}>25\textrm{ GeV}$ &  $p_{T}^{l}>27\textrm{ GeV}$ \\
\hline
$m_{ll}$ &  $66 \textrm{ GeV} < m_{ll} < 116 \textrm{ GeV}$ & $66 \textrm{ GeV} < m_{ll} <116 \textrm{ GeV}$ \\
\hline
\end{tabular}
\caption{Comparison of fiducial phase space between low-$\mu$ and high-$\mu$ p$_{\mathrm{T}}^{Z}$ measurements.}
\label{tab:fiducial}
\end{table}

\begin{figure}[h]
\centering
\includegraphics[width=.69\textwidth]{figure/ZpT_highmu_compatibility/Correction_2GeVBin.pdf}
\caption{Correction factor as a function of $p_{T}$ accounting for the different fiducial phase spaces in the low-$\mu$ and high-$\mu$ analyses.}
\label{f:Correction}
\end{figure}


To calculate the compatibility between the high-$\mu$ and low-$\mu$ measurements, a bin-by-bin $\chi^2$ can be calculated using \Eqn{\ref{eq:generalchi2}}:

\begin{equation}
\chi^2 = \sum_{i} \frac{(p_{\mathrm{T}, low-\mu}^{i} - p_{\mathrm{T}, high-\mu}^{i})^2}{\sigma^2 (p_{\mathrm{T}, low-\mu}^{i} )+\sigma^2 (p_{\mathrm{T},
high-\mu}^{i} )+2\times Corr\left(\sigma (p_{\mathrm{T}, high-\mu}^{i} ),\sigma (p_{\mathrm{T}, high-\mu}^{i} )\right)}
\label{eq:generalchi2}
\end{equation}

Generally, some of the uncertainties such as the lepton performance systematics will be correlated.
However, in the low-$\mu$ analysis calibrations and corrections are mostly derived directly using the low-$\mu$ dataset itself, (details in~Refs \cite{Xu:2657152}, \cite{Sydorenko:2657116}) while in the high-$\mu$ analysis, the calibrations and corrections are derived using standard ATLAS high-$\mu$ dataset, thus the uncertainties are treated approximately as uncorrelated.
Therefore, the combined uncertainty equals to the sum of uncertainties from each analysis in quadrature.
The bin-by-bin $\chi^2$ is calculated using \Eqn{\ref{eq:chi2highmu}}:
\begin{equation}
\chi^2 = \sum_{i} \frac{(p_{\mathrm{T}, low-\mu}^{i} - p_{\mathrm{T}, high-\mu}^{i})^2}{\sigma^2 (p_{\mathrm{T}, low-\mu}^{i} )+\sigma^2 (p_{\mathrm{T}, high-\mu}^{i} )} \textrm{,}
\label{eq:chi2highmu}
\end{equation}
where the uncertainties of the low-$\mu$ and high-$\mu$ measurements are the quadratic sums of their statistical and systematic uncertainties as is show in \Eqn{\ref{eq:sigmalowmu} and \ref{eq:sigmahighmu}}.

\begin{equation}
\sigma^2 (p_{\mathrm{T}, low-\mu}^{i}) = \sqrt{ \sigma^2(low-\mu, stat.) + \sigma^2(low-\mu, syst.) }
\label{eq:sigmalowmu}
\end{equation}

\begin{equation}
\sigma^2 (p_{\mathrm{T}, high-\mu}^{i}) = \sqrt{ \sigma^2(high-\mu, stat.) + \sigma^2(high-\mu, syst.) }
\label{eq:sigmahighmu}
\end{equation}

The $p_{T}^{Z}$ distribution must also be re-binned in order to compare the two measurements because different binning was used in each analysis.
The $p_{T}^{Z}$ bins in the high-$\mu$ analysis are
$$[0, 2, 4, 6, 8, 10, 12, 14, 16, 18, 20, 22.5, 25, 27.5, 30, 33, 36, 39, 42, 45, 48, 51, 54, 57, 61, 65, 70, 75, 80, 85,$$
$$95, 105, 125, 150, 175, 200, 250, 300, 350, 400, 470, 550, 650, 900, 2500]$$
while for the low-$\mu$ analysis, the binning is
$$[0, 2, 4, 6, 8, 10, 12, 14, 17, 20, 23, 26, 29, 33, 37, 41, 47, 53, 60, 70, 80, 100, 150, 200, 600]$$

To account for this difference, the high-$\mu$ $p_{T}^{Z}$ distribution is re-binned to the low-$\mu$ binning.
Whenever the rebinning is not allowed, the bin values in the new re-binned comparison histograms are weighted by bin width,
and the relative uncertainty is the error of the bin in old binning which contains the center of the low-$\mu$ binning.
E.G. while re-binning 14,16,18,20 (in high-mu binning)  to  14,17,20 (in low-mu binning):

$$\frac{d\sigma}{dp_\mathrm{T}}_{bin (14,17)}^{high-\mu, rebinned} = \frac{ \frac{d\sigma}{dp_\mathrm{T}}_{bin(14,16)}^{high-\mu} \times 2GeV + \frac{d\sigma}{dp_\mathrm{T}}_{bin(16,18)}^{high-\mu} \times 1GeV }{3GeV}$$

The relative error is taken from the bin where the center of new bin locates. E.G. for low-$\mu$ bin (14,17) GeV, the center is 15.5 GeV, which is in the high-$\mu$ bin (14,16) GeV, the relative error of bin (14,16)~GeV for the re-binned bin (14,17)~GeV.

This rebinning method is only approximate as it assumes the cross-section to be flat within a bin, but allows to consider more bins than just taking the smallest common binning, which would force us to $e.g.$ have a big bin between 14 and 20~\GeV.

Comparisons between high-$\mu$ and low-$\mu$ ~\ref{sec:physcorr} measurements of $p_{T}^{Z}$ distributions are shown in \Fig{\ref{f:CompatibilityHighLowMuRew}}. Good agreement is observed between the high-$\mu$ measurement in cyan and the low-$\mu$ measurement in black. In most of the bins, the ratio is within 1~$\sigma$.
$\chi^2$ in the $Z\rightarrow ee$ channel is measured to be $\chi^2$/dof =  0.676 and $\chi^2$/dof = 0.96 in the $Z\rightarrow \mu\mu$ channel~(\Tab{\ref{tab:chi2highmuRew}}).

\begin{figure}[h]
  \centering
  \includegraphics[width=.45\textwidth]{figure/ZpT_highmu_compatibility/XSec_Compare_highmu_13TeV_ee_2GeVBin_Rew.pdf}
   \includegraphics[width=.45\textwidth]{figure/ZpT_highmu_compatibility/XSec_Compare_highmu_13TeV_mumu_2GeVBin_Rew.pdf}
  \caption{Compatibility of the $p_{T}^{Z}$ measurement between the high-$\mu$ and low-$\mu$ analyses at the unfolded level. The left plot is the $Z\rightarrow ee$ channel and the right plot is the $Z\rightarrow \mu\mu$ channel.}
    \label{f:CompatibilityHighLowMuRew}
\end{figure}

\begin{table}[h]
 \centering
\begin{tabular}{|c|c|c|}
\hline
  & $Z\rightarrow ee$ & $Z\rightarrow \mu\mu$\\
 \hline
 $\chi^2$ & 15.55 & 22.09 \\
 \hline
$\chi^2/$ dof & 0.676 & 0.96\\
\hline
\end{tabular}
\caption{$\chi^2$ results of $p_{T}^{Z}$ measurements between the high-$\mu$ and low-$\mu$  analyses.}
\label{tab:chi2highmuRew}
\end{table}

\clearpage
\subsubsection{Compatibility between the dilepton method and hadronic recoil method for measuring $p_{T}^{Z}$}
\label{ssec:pTZcompatibility}

Here we present studies on the compatibility of the unfolded p$_\mathrm{T}^Z$ spectrums using the dilepton system (\ptdilep) and the hadronic recoil system (\ut). Unless explicitly stated, all studies are based on unfolded results using the \pt-reweighted MC (as described in \ref{sec:physcorrWpTrew}). These studies help validate the $p_{\mathrm{T}}^{W}$ measurement which is measured indirectly through the hadronic recoil method because the dilepton method is invalid due to the non-detection of the W-decayed neutrino. Unfolded results in \Sect{\ref{sec:Zresult}} are compared using unfolding iterations that have been preliminarily optimized by minimizing the overall uncertainty.

To re-introduce the unfolded \ptdilep and \ut distributions, we first show their normalized differential cross-sections on the same plot compared to the nominal Powheg+Pythia8 MC. Figure \ref{f:unf_pTuT_13coarse} shows these normalized differential cross-sections for both \Zee and \Zmm at 13 TeV, while Figure \ref{f:unf_pTuT_5coarse} shows the corresponding plots at 5 TeV. These individual distributions are identical to what is shown in Figures \ref{f:MCData_Z_13} and \ref{f:MCData_Z_5} except that the alternative MC generator predictions have been removed and the \ptdilep and \ut distributions are shown on the same plot.

\begin{figure}[h]
\centering
\subfloat[\Zee]{\includegraphics[width=.49\textwidth]{figure/Z_Plots_unfolding_results/13TeV_coarse/MC_Data_v20201127_both_Zee_13TeV_5GeVbins_5uTiters.pdf}\label{f:unf_both_Zee13coarse}}
\subfloat[\Zmm]{\includegraphics[width=.49\textwidth]{figure/Z_Plots_unfolding_results/13TeV_coarse/MC_Data_v20201127_both_Zmumu_13TeV_5GeVbins_5uTiters.pdf}\label{f:unf_both_Zmm13coarse}}
\caption{Unfolded \ptz\ differential cross-section using either \ptdilep (black) or \ut (blue) for the $\sqrt{s} = 13$~\TeV\ dataset. The nominal Powheg+Pythia8 Monte Carlo generator prediction, without the \pt-correction, is shown in red. The differential cross-section has been normalized by the total cross-section. The unfolded data and MC distributions are shown in the upper panel and the lower panel shows the ratio of the various distributions to the data \ptdilep\ unfolded result. The total statistical and systematic uncertainties are included in the data error bands.}\label{f:unf_pTuT_13coarse}
\end{figure}

\begin{figure}[h]
\centering
\subfloat[]{\includegraphics[width=.49\textwidth]{figure/Z_Plots_unfolding_results/5TeV_coarse/MC_Data_v20201127_both_Zee_5TeV_5GeVbins_2uTiters.pdf}\label{f:unf_both_Zee5coarse}}
\subfloat[]{\includegraphics[width=.49\textwidth]{figure/Z_Plots_unfolding_results/5TeV_coarse/MC_Data_v20201127_both_Zmumu_5TeV_5GeVbins_2uTiters.pdf}\label{f:unf_both_Zmm5coarse}}
\caption{Unfolded \ptz\ differential cross-section using either \ptdilep (black) or \ut (blue) for the $\sqrt{s} = 5$~\TeV\ dataset. The nominal Powheg+Pythia8 Monte Carlo generator prediction, without the \pt-correction, is shown in red. The differential cross-section has been normalized by the total cross-section. The unfolded data and MC distributions are shown in the upper panel and the lower panel shows the ratio of the various distributions to the data \ptdilep\ unfolded result. The total statistical and systematic uncertainties are included in the data error bands.}\label{f:unf_pTuT_5coarse}
\end{figure}

Direct comparisons of p$_\mathrm{T}^{Z}$ (ll) and p$_\mathrm{T}^{Z}$ (u$_\mathrm{T}$) with residuals, which concerns only uncorrelated statistical uncertainties (defined in \Eqn{\ref{eq:residuals}}), are shown in  \Fig{\ref{f:DirCompTuTRew}}.
In these plots, only statistical uncertainties are included without correlations.
Correlations between the dilepton method and the hadronic recoil method are discussed in the following section.

\begin{equation}
Residuals = \sum_{i} \frac{(p_\mathrm{T}^{Z} (ll) - p_\mathrm{T}^{Z} (u_\mathrm{T}))^2}{\sigma_{stat.}^2 (p_\mathrm{T}^{Z} (ll) )+\sigma_{stat.}^2 (p_\mathrm{T}^{Z} (u_\mathrm{T}) )}
\label{eq:residuals}
\end{equation}

\begin{figure}[h]
  \centering
  \includegraphics[width=.4\textwidth]{figure/ZpTuT_compatibility_Rew_v20210511/pTuT_5TeV_ee_acc_5GeVBin_2uTiters.pdf}
  \includegraphics[width=.4\textwidth]{figure/ZpTuT_compatibility_Rew_v20210511/pTuT_5TeV_mumu_acc_5GeVBin_2uTiters.pdf}
  \includegraphics[width=.4\textwidth]{figure/ZpTuT_compatibility_Rew_v20210511/pTuT_13TeV_ee_acc_5GeVBin_5uTiters.pdf}
  \includegraphics[width=.4\textwidth]{figure/ZpTuT_compatibility_Rew_v20210511/pTuT_13TeV_mumu_acc_5GeVBin_5uTiters.pdf}
  \caption{Residuals of the $p_{\mathrm{T}}^{Z}$ measurements comparing the $p_\mathrm{T}^{Z}$ (ll) and  $p_\mathrm{T}^{Z}$ (u$_\mathrm{T}$) using only statistical uncertainty. The uncertainty is (wrongly) considered as uncorrelated between the methods.}
  \label{f:DirCompTuTRew}
\end{figure}

The uncertainties in unfolded  $p_\mathrm{T}^{Z}$ (ll) and $p_\mathrm{T}^{Z}$ (u$_\mathrm{T}$) are highly correlated so the covariances between the two measurements must be taken into account.
A general definition of the covariance matrix (denoted as $C$) is given by \Eqn{\ref{eq:Covariance}}:

\begin{equation}
C_{i,j} = \sum_k^{N_{NP}} (p_{\mathrm{T, toys[k]}}^i (ll) - p_{\mathrm{T, toys[k]}}^i (u_\mathrm{T})) \times  (p_{\mathrm{T, toys[k]}}^j (ll) - p_{\mathrm{T, toys[k]}}^j (u_\mathrm{T}))\textrm{,}
\label{eq:Covariance}
\end{equation}
where $i$ and $j$ are bins in $p_{\mathrm{T}}$, and $k$ loops over all nuisance parameters.
Statistical uncertainty dominates in the analysis because of the small dataset~(335.9 pb$^{-1}$ at 13~TeV and 256.8 pb$^{-1}$ at 5~TeV),
which is the main contribution to the covariance.
The bootstrap method is used to estimate the statistical uncertainty, in which 1000 toy distributions are generated, fluctuating around the nominal $p_\mathrm{T}^{Z}$ (ll) - $p_\mathrm{T}^{Z}$ (u$_\mathrm{T}$) map following Poisson distributions.
To include the $p_\mathrm{T}^{Z}$ (ll) - $p_\mathrm{T}^{Z}$ (u$_\mathrm{T}$)  correlations and bin-to-bin correlations at the same time, a double-subtraction must be included in the covariance matrix definition \Eqn{\ref{eq:CovarianceStat}}:

\begin{equation}
C_{i,j}^{stat.} =\frac{1}{N} \sum_k [(p_{\mathrm{T,toy[k]}}^{i} (ll)-p_{\mathrm{T}}^{i}(ll))-(p_{\mathrm{T,toy[k]}}^{i}(u_T)-p_{\mathrm{T}}^{i}(u_T))] \times [(p_{\mathrm{T,toy[k]}}^{j} (ll) - p_{\mathrm{T}}^{j}(ll))-(p_{\mathrm{T,toy[k]}}^{j} (u_T) -p_{\mathrm{T}}^{j})(u_T)]\textrm{.}
\label{eq:CovarianceStat}
\end{equation}
Here, $i$ and $j$ are the bins in $p_{\mathrm{T}}$, and $k$ loops over all bootstrap toys. Physically, the unfolded nominal $p_{\mathrm{T}}^{Z}(ll)$ and $p_{\mathrm{T}}^{Z}(u_T)$ distributions are equivalent, because they are both methods of measuring the truth $p_{\mathrm{T}}^{Z}$.
At this point, $p_{\mathrm{T}}^{Z}(ll)$ and $p_{\mathrm{T}}^{Z}(u_T)$ cancel each other, and the double-subtraction definition is equivalent to the original definition \Eqn{\ref{eq:Covariance}}.
Systematic covariance matrices are built similarly, as can be seen in \Eqn{\ref{eq:CovarianceSyst}}:

\begin{equation}
C_{i,j}^{syst.} = \sum_k [(p_{\mathrm{T,NP[k]}}^{i} (ll)-p_{\mathrm{T}}^{i} (ll))-(p_{\mathrm{T,NP[k]}}^{i} (u_T)-p_{\mathrm{T}}^{i} (u_T))] \times  [(p_{\mathrm{T,NP[k]}}^{j}(ll)-p_{\mathrm{T}}^{j}(ll))-(p_{\mathrm{T,NP[k]}}^{j} (u_T)-p_{\mathrm{T}}^{j}(u_T))]\textrm{.}
\label{eq:CovarianceSyst}
\end{equation}



\clearpage

\paragraph{Covariance Matrices}


Covariance matrices from each nuisance parameter are summed directly, and the total covariance matrix is the statistical covariance matrix plus the systematic covariance matrix.
Overall uncertainty breakdown plots are shown in a previous section (\ref{ss:ZpT_all_unc}). The leading contributions in the matrices are the statistical uncertainties and the bias uncertainty.
Covariance matrices in each channel and at each center-of-mass energy are shown in~Figs \ref{f:CovMatrix5TeVZeeRew}, \ref{f:CovMatrix5TeVZmumuRew}, \ref{f:CovMatrix13TeVZeeRew}, \ref{f:CovMatrix13TeVZmumuRew}.
In the covariance matrices of statistics, unfolded p$_{\mathrm{T}}^{Z}(ll)$ and p$_\mathrm{T}^{Z}(u_T)$ correlate along the diagonal,
and anti-correlate in the bins neighbouring the diagonal.
The unfolding bias uncertainty contributes significantly at low-$p_\mathrm{T}$ because of $p_\mathrm{T}$ modeling.

\begin{figure}[h]
\centering
\subfloat[]{\includegraphics[width=.2\textwidth]{figure/ZpTuT_compatibility_Rew_v20210511/CovMatrix_dataStat_5TeV_ee_5GeVbin_2uTiters_acc.pdf}}
\subfloat[]{\includegraphics[width=.2\textwidth]{figure/ZpTuT_compatibility_Rew_v20210511/CovMatrix_MCStat_5TeV_ee_5GeVbin_2uTiters_acc.pdf}}
\subfloat[]{\includegraphics[width=.2\textwidth]{figure/ZpTuT_compatibility_Rew_v20210511/CovMatrix_recoil_5TeV_ee_5GeVbin_2uTiters_acc.pdf}}
\subfloat[]{\includegraphics[width=.2\textwidth]{figure/ZpTuT_compatibility_Rew_v20210511/CovMatrix_bias_5TeV_ee_5GeVbin_2uTiters_acc.pdf}}
\subfloat[]{\includegraphics[width=.2\textwidth]{figure/ZpTuT_compatibility_Rew_v20210511/CovMatrix_elcalib_5TeV_ee_5GeVbin_2uTiters_acc.pdf}}\\
\subfloat[]{\includegraphics[width=.2\textwidth]{figure/ZpTuT_compatibility_Rew_v20210511/CovMatrix_ElID_5TeV_ee_5GeVbin_2uTiters_acc.pdf}}
\subfloat[]{\includegraphics[width=.2\textwidth]{figure/ZpTuT_compatibility_Rew_v20210511/CovMatrix_ElReco_5TeV_ee_5GeVbin_2uTiters_acc.pdf}}
\subfloat[]{\includegraphics[width=.2\textwidth]{figure/ZpTuT_compatibility_Rew_v20210511/CovMatrix_ElTrig_5TeV_ee_5GeVbin_2uTiters_acc.pdf}}
\subfloat[]{\includegraphics[width=.2\textwidth]{figure/ZpTuT_compatibility_Rew_v20210511/CovMatrix_ElIso_5TeV_ee_5GeVbin_2uTiters_acc.pdf}}
\caption{Covariance matrices for each uncertainty in the $Z\rightarrow ee$ channel at 5~TeV. (a) data statistics. (b) MC statistics. (c) recoil calibtration. (d) unfolding bias. (e) electron calibration. (f)Electron ID efficiency. (g)Electron reconstruction efficiency. (h)Electron trigger efficiency (i) Electron isolation efficiency }
\label{f:CovMatrix5TeVZeeRew}
\end{figure}

\begin{figure}[h]
\centering
\subfloat[]{\includegraphics[width=.2\textwidth]{figure/ZpTuT_compatibility_Rew_v20210511/CovMatrix_dataStat_5TeV_mumu_5GeVbin_2uTiters_acc.pdf}}
\subfloat[]{\includegraphics[width=.2\textwidth]{figure/ZpTuT_compatibility_Rew_v20210511/CovMatrix_MCStat_5TeV_mumu_5GeVbin_2uTiters_acc.pdf}}
\subfloat[]{\includegraphics[width=.2\textwidth]{figure/ZpTuT_compatibility_Rew_v20210511/CovMatrix_recoil_5TeV_mumu_5GeVbin_2uTiters_acc.pdf}}
\subfloat[]{\includegraphics[width=.2\textwidth]{figure/ZpTuT_compatibility_Rew_v20210511/CovMatrix_bias_5TeV_mumu_5GeVbin_2uTiters_acc.pdf}}
\subfloat[]{\includegraphics[width=.2\textwidth]{figure/ZpTuT_compatibility_Rew_v20210511/CovMatrix_mucalib_5TeV_mumu_5GeVbin_2uTiters_acc.pdf}}\\
\subfloat[]{\includegraphics[width=.2\textwidth]{figure/ZpTuT_compatibility_Rew_v20210511/CovMatrix_MuReco_5TeV_mumu_5GeVbin_2uTiters_acc.pdf}}
\subfloat[]{\includegraphics[width=.2\textwidth]{figure/ZpTuT_compatibility_Rew_v20210511/CovMatrix_MuTrig_5TeV_mumu_5GeVbin_2uTiters_acc.pdf}}
\subfloat[]{\includegraphics[width=.2\textwidth]{figure/ZpTuT_compatibility_Rew_v20210511/CovMatrix_MuIso_5TeV_mumu_5GeVbin_2uTiters_acc.pdf}}
\subfloat[]{\includegraphics[width=.2\textwidth]{figure/ZpTuT_compatibility_Rew_v20210511/CovMatrix_MuTTVA_5TeV_mumu_5GeVbin_2uTiters_acc.pdf}}
\caption{Covariance matrices for each uncertainty in the $Z\rightarrow \mu\mu$ channel at 5~TeV. (a) data statistics. (b) MC statistics. (c) recoil calibtration. (d) unfolding bias. (e) muon calibration. (f)muon reconstruction efficiency. (g)muon trigger efficiency. (h)muon isolation efficiency (i) muon TTVA efficiency }
\label{f:CovMatrix5TeVZmumuRew}
\end{figure}

\begin{figure}[h]
\centering
\subfloat[]{\includegraphics[width=.2\textwidth]{figure/ZpTuT_compatibility_Rew_v20210511/CovMatrix_dataStat_13TeV_ee_5GeVbin_5uTiters_acc.pdf}}
\subfloat[]{\includegraphics[width=.2\textwidth]{figure/ZpTuT_compatibility_Rew_v20210511/CovMatrix_MCStat_13TeV_ee_5GeVbin_5uTiters_acc.pdf}}
\subfloat[]{\includegraphics[width=.2\textwidth]{figure/ZpTuT_compatibility_Rew_v20210511/CovMatrix_recoil_13TeV_ee_5GeVbin_5uTiters_acc.pdf}}
\subfloat[]{\includegraphics[width=.2\textwidth]{figure/ZpTuT_compatibility_Rew_v20210511/CovMatrix_bias_13TeV_ee_5GeVbin_5uTiters_acc.pdf}}
\subfloat[]{\includegraphics[width=.2\textwidth]{figure/ZpTuT_compatibility_Rew_v20210511/CovMatrix_elcalib_13TeV_ee_5GeVbin_5uTiters_acc.pdf}}\\
\subfloat[]{\includegraphics[width=.2\textwidth]{figure/ZpTuT_compatibility_Rew_v20210511/CovMatrix_ElID_13TeV_ee_5GeVbin_5uTiters_acc.pdf}}
\subfloat[]{\includegraphics[width=.2\textwidth]{figure/ZpTuT_compatibility_Rew_v20210511/CovMatrix_ElReco_13TeV_ee_5GeVbin_5uTiters_acc.pdf}}
\subfloat[]{\includegraphics[width=.2\textwidth]{figure/ZpTuT_compatibility_Rew_v20210511/CovMatrix_ElTrig_13TeV_ee_5GeVbin_5uTiters_acc.pdf}}
\subfloat[]{\includegraphics[width=.2\textwidth]{figure/ZpTuT_compatibility_Rew_v20210511/CovMatrix_ElIso_13TeV_ee_5GeVbin_5uTiters_acc.pdf}}
\caption{Covariance matrices for each uncertainty in the $Z\rightarrow ee$ channel at 13~TeV. (a) data statistics. (b) MC statistics. (c) recoil calibtration. (d) unfolding bias. (e) electron calibration. (f)Electron ID efficiency. (g)Electron reconstruction efficiency. (h)Electron trigger efficiency (i) Electron isolation efficiency }
\label{f:CovMatrix13TeVZeeRew}
\end{figure}

\begin{figure}[h]
\centering
\subfloat[]{\includegraphics[width=.2\textwidth]{figure/ZpTuT_compatibility_Rew_v20210511/CovMatrix_dataStat_13TeV_mumu_5GeVbin_5uTiters_acc.pdf}}
\subfloat[]{\includegraphics[width=.2\textwidth]{figure/ZpTuT_compatibility_Rew_v20210511/CovMatrix_MCStat_13TeV_mumu_5GeVbin_5uTiters_acc.pdf}}
\subfloat[]{\includegraphics[width=.2\textwidth]{figure/ZpTuT_compatibility_Rew_v20210511/CovMatrix_recoil_13TeV_mumu_5GeVbin_5uTiters_acc.pdf}}
\subfloat[]{\includegraphics[width=.2\textwidth]{figure/ZpTuT_compatibility_Rew_v20210511/CovMatrix_bias_13TeV_mumu_5GeVbin_5uTiters_acc.pdf}}
\subfloat[]{\includegraphics[width=.2\textwidth]{figure/ZpTuT_compatibility_Rew_v20210511/CovMatrix_mucalib_13TeV_mumu_5GeVbin_5uTiters_acc.pdf}}\\
\subfloat[]{\includegraphics[width=.2\textwidth]{figure/ZpTuT_compatibility_Rew_v20210511/CovMatrix_MuReco_13TeV_mumu_5GeVbin_5uTiters_acc.pdf}}
\subfloat[]{\includegraphics[width=.2\textwidth]{figure/ZpTuT_compatibility_Rew_v20210511/CovMatrix_MuTrig_13TeV_mumu_5GeVbin_5uTiters_acc.pdf}}
\subfloat[]{\includegraphics[width=.2\textwidth]{figure/ZpTuT_compatibility_Rew_v20210511/CovMatrix_MuIso_13TeV_mumu_5GeVbin_5uTiters_acc.pdf}}
\subfloat[]{\includegraphics[width=.2\textwidth]{figure/ZpTuT_compatibility_Rew_v20210511/CovMatrix_MuTTVA_13TeV_mumu_5GeVbin_5uTiters_acc.pdf}}
\caption{Covariance matrices for each uncertainty in the $Z\rightarrow \mu\mu$ channel at 13~TeV. (a) data statistics. (b) MC statistics. (c) recoil calibtration. (d) unfolding bias. (e) muon calibration. (f)muon reconstruction efficiency. (g)muon trigger efficiency. (h)muon isolation efficiency (i) muon TTVA efficiency.}
\label{f:CovMatrix13TeVZmumuRew}
\end{figure}


\paragraph{Correlation Matrices}
	Correlation matrices defined as the normalization of Covariance matrices with the diagonal elements : $Corr_{i,j} = Cov(i,j)/ \sqrt{(Cov(i,i)^2 + Cov(j,j)^2)}$, are shown in~Figs \ref{f:CorrMatrix5TeVZeeRew}, \ref{f:CorrMatrix5TeVZmumuRew}, \ref{f:CorrMatrix13TeVZeeRew}, \ref{f:CorrMatrix13TeVZmumuRew}.

\begin{figure}[h]
\centering
\subfloat[]{\includegraphics[width=.2\textwidth]{figure/ZpTuT_compatibility_Rew_v20210511/CorrMatrix_dataStat_5TeV_ee_5GeVbin2uTiters_acc.pdf}}
\subfloat[]{\includegraphics[width=.2\textwidth]{figure/ZpTuT_compatibility_Rew_v20210511/CorrMatrix_MCStat_5TeV_ee_5GeVbin2uTiters_acc.pdf}}
\subfloat[]{\includegraphics[width=.2\textwidth]{figure/ZpTuT_compatibility_Rew_v20210511/CorrMatrix_recoil_5TeV_ee_5GeVbin2uTiters_acc.pdf}}
\subfloat[]{\includegraphics[width=.2\textwidth]{figure/ZpTuT_compatibility_Rew_v20210511/CorrMatrix_bias_5TeV_ee_5GeVbin2uTiters_acc.pdf}}
\subfloat[]{\includegraphics[width=.2\textwidth]{figure/ZpTuT_compatibility_Rew_v20210511/CorrMatrix_elcalib_5TeV_ee_5GeVbin2uTiters_acc.pdf}}\\
\subfloat[]{\includegraphics[width=.2\textwidth]{figure/ZpTuT_compatibility_Rew_v20210511/CorrMatrix_ElID_5TeV_ee_5GeVbin2uTiters_acc.pdf}}
\subfloat[]{\includegraphics[width=.2\textwidth]{figure/ZpTuT_compatibility_Rew_v20210511/CorrMatrix_ElReco_5TeV_ee_5GeVbin2uTiters_acc.pdf}}
\subfloat[]{\includegraphics[width=.2\textwidth]{figure/ZpTuT_compatibility_Rew_v20210511/CorrMatrix_ElTrig_5TeV_ee_5GeVbin2uTiters_acc.pdf}}
\subfloat[]{\includegraphics[width=.2\textwidth]{figure/ZpTuT_compatibility_Rew_v20210511/CorrMatrix_ElIso_5TeV_ee_5GeVbin2uTiters_acc.pdf}}
\caption{Correlation matrices for each uncertainty in the $Z\rightarrow ee$ channel at 5~TeV. (a) data statistics. (b) MC statistics. (c) recoil calibtration. (d) unfolding bias. (e) electron calibration. (f)Electron ID efficiency. (g)Electron reconstruction efficiency. (h)Electron trigger efficiency (i) Electron isolation efficiency }
\label{f:CorrMatrix5TeVZeeRew}
\end{figure}

\begin{figure}[h]
\centering
\subfloat[]{\includegraphics[width=.2\textwidth]{figure/ZpTuT_compatibility_Rew_v20210511/CorrMatrix_dataStat_5TeV_mumu_5GeVbin2uTiters_acc.pdf}}
\subfloat[]{\includegraphics[width=.2\textwidth]{figure/ZpTuT_compatibility_Rew_v20210511/CorrMatrix_MCStat_5TeV_mumu_5GeVbin2uTiters_acc.pdf}}
\subfloat[]{\includegraphics[width=.2\textwidth]{figure/ZpTuT_compatibility_Rew_v20210511/CorrMatrix_recoil_5TeV_mumu_5GeVbin2uTiters_acc.pdf}}
\subfloat[]{\includegraphics[width=.2\textwidth]{figure/ZpTuT_compatibility_Rew_v20210511/CorrMatrix_bias_5TeV_mumu_5GeVbin2uTiters_acc.pdf}}
\subfloat[]{\includegraphics[width=.2\textwidth]{figure/ZpTuT_compatibility_Rew_v20210511/CorrMatrix_mucalib_5TeV_mumu_5GeVbin2uTiters_acc.pdf}}\\
\subfloat[]{\includegraphics[width=.2\textwidth]{figure/ZpTuT_compatibility_Rew_v20210511/CorrMatrix_MuReco_5TeV_mumu_5GeVbin2uTiters_acc.pdf}}
\subfloat[]{\includegraphics[width=.2\textwidth]{figure/ZpTuT_compatibility_Rew_v20210511/CorrMatrix_MuTrig_5TeV_mumu_5GeVbin2uTiters_acc.pdf}}
\subfloat[]{\includegraphics[width=.2\textwidth]{figure/ZpTuT_compatibility_Rew_v20210511/CorrMatrix_MuIso_5TeV_mumu_5GeVbin2uTiters_acc.pdf}}
\subfloat[]{\includegraphics[width=.2\textwidth]{figure/ZpTuT_compatibility_Rew_v20210511/CorrMatrix_MuTTVA_5TeV_mumu_5GeVbin2uTiters_acc.pdf}}
\caption{Correlation matrices for each uncertainty in the $Z\rightarrow \mu\mu$ channel at 5~TeV. (a) data statistics. (b) MC statistics. (c) recoil calibtration. (d) unfolding bias. (e) muon calibration. (f)muon reconstruction efficiency. (g)muon trigger efficiency. (h)muon isolation efficiency (i) muon TTVA efficiency }
\label{f:CorrMatrix5TeVZmumuRew}
\end{figure}

\begin{figure}[h]
\centering
\subfloat[]{\includegraphics[width=.2\textwidth]{figure/ZpTuT_compatibility_Rew_v20210511/CorrMatrix_dataStat_13TeV_ee_5GeVbin5uTiters_acc.pdf}}
\subfloat[]{\includegraphics[width=.2\textwidth]{figure/ZpTuT_compatibility_Rew_v20210511/CorrMatrix_MCStat_13TeV_ee_5GeVbin5uTiters_acc.pdf}}
\subfloat[]{\includegraphics[width=.2\textwidth]{figure/ZpTuT_compatibility_Rew_v20210511/CorrMatrix_recoil_13TeV_ee_5GeVbin5uTiters_acc.pdf}}
\subfloat[]{\includegraphics[width=.2\textwidth]{figure/ZpTuT_compatibility_Rew_v20210511/CorrMatrix_bias_13TeV_ee_5GeVbin5uTiters_acc.pdf}}
\subfloat[]{\includegraphics[width=.2\textwidth]{figure/ZpTuT_compatibility_Rew_v20210511/CorrMatrix_elcalib_13TeV_ee_5GeVbin5uTiters_acc.pdf}}\\
\subfloat[]{\includegraphics[width=.2\textwidth]{figure/ZpTuT_compatibility_Rew_v20210511/CorrMatrix_ElID_13TeV_ee_5GeVbin5uTiters_acc.pdf}}
\subfloat[]{\includegraphics[width=.2\textwidth]{figure/ZpTuT_compatibility_Rew_v20210511/CorrMatrix_ElReco_13TeV_ee_5GeVbin5uTiters_acc.pdf}}
\subfloat[]{\includegraphics[width=.2\textwidth]{figure/ZpTuT_compatibility_Rew_v20210511/CorrMatrix_ElTrig_13TeV_ee_5GeVbin5uTiters_acc.pdf}}
\subfloat[]{\includegraphics[width=.2\textwidth]{figure/ZpTuT_compatibility_Rew_v20210511/CorrMatrix_ElIso_13TeV_ee_5GeVbin5uTiters_acc.pdf}}
\caption{Correlation matrices for each uncertainty in the $Z\rightarrow ee$ channel at 13~TeV. (a) data statistics. (b) MC statistics. (c) recoil calibtration. (d) unfolding bias. (e) electron calibration. (f)Electron ID efficiency. (g)Electron reconstruction efficiency. (h)Electron trigger efficiency (i) Electron isolation efficiency }
\label{f:CorrMatrix13TeVZeeRew}
\end{figure}

\begin{figure}[h]
\centering
\subfloat[]{\includegraphics[width=.2\textwidth]{figure/ZpTuT_compatibility_Rew_v20210511/CorrMatrix_dataStat_13TeV_mumu_5GeVbin5uTiters_acc.pdf}}
\subfloat[]{\includegraphics[width=.2\textwidth]{figure/ZpTuT_compatibility_Rew_v20210511/CorrMatrix_MCStat_13TeV_mumu_5GeVbin5uTiters_acc.pdf}}
\subfloat[]{\includegraphics[width=.2\textwidth]{figure/ZpTuT_compatibility_Rew_v20210511/CorrMatrix_recoil_13TeV_mumu_5GeVbin5uTiters_acc.pdf}}
\subfloat[]{\includegraphics[width=.2\textwidth]{figure/ZpTuT_compatibility_Rew_v20210511/CorrMatrix_bias_13TeV_mumu_5GeVbin5uTiters_acc.pdf}}
\subfloat[]{\includegraphics[width=.2\textwidth]{figure/ZpTuT_compatibility_Rew_v20210511/CorrMatrix_mucalib_13TeV_mumu_5GeVbin5uTiters_acc.pdf}}\\
\subfloat[]{\includegraphics[width=.2\textwidth]{figure/ZpTuT_compatibility_Rew_v20210511/CorrMatrix_MuReco_13TeV_mumu_5GeVbin5uTiters_acc.pdf}}
\subfloat[]{\includegraphics[width=.2\textwidth]{figure/ZpTuT_compatibility_Rew_v20210511/CorrMatrix_MuTrig_13TeV_mumu_5GeVbin5uTiters_acc.pdf}}
\subfloat[]{\includegraphics[width=.2\textwidth]{figure/ZpTuT_compatibility_Rew_v20210511/CorrMatrix_MuIso_13TeV_mumu_5GeVbin5uTiters_acc.pdf}}
\subfloat[]{\includegraphics[width=.2\textwidth]{figure/ZpTuT_compatibility_Rew_v20210511/CorrMatrix_MuTTVA_13TeV_mumu_5GeVbin5uTiters_acc.pdf}}
\caption{Correlation matrices for each uncertainty in the $Z\rightarrow \mu\mu$ channel at 13~TeV.
(a) data statistics. (b) MC statistics. (c) recoil calibtration. (d) unfolding bias. (e) muon calibration. (f)muon reconstruction efficiency. (g)muon trigger efficiency. (h)muon isolation efficiency (i) muon TTVA efficiency.}
\label{f:CorrMatrix13TeVZmumuRew}
\end{figure}


\clearpage



%%\paragraph{$\chi^2$ results using reweighted MC}
The calculated $\chi^2$ values are shown in \Tab{\ref{tab:chi2Rew}}. The $\chi^2$ results at 13~TeV are better than those numbers before reweighting.
The degree of freedom (dof) is $N_{bins} - 1 =14$ because there is one constrain additional constraint: the sums of $p_{\mathrm{T}}^{ll}$ and $u_\mathrm{T}$ must be equal:
$\sum_{i} p_{\mathrm{T}}^{ll}[bin_{i}] = \sum_{i} u_{\mathrm{T}}[bin_{i}]$.

\begin{table}[h]
\centering
\begin{tabular}{|c|c|c|c|c|}
\hline
$\chi^2$ & 5~TeV $Z\rightarrow ee$ & 5~TeV $Z\rightarrow \mu\mu$ & 13~TeV $Z\rightarrow ee$ & 13~TeV $Z\rightarrow \mu\mu$\\
\hline
Stat. Uncertainty & 31.7 & 16.4 & 23.1 & 73.2 \\
\hline
Stat. + Sys. Uncertainty & 30.0& 14.0 & 17.1 & 39.1 \\
\hline
$\chi^2$ /dof (Stat.+Syst.) & 2.14 &1.0 & 1.22 & 2.79 \\
\hline
 {\color{blue} $\chi^2$ /dof (without reweighting) }& 1.29 &1.04 & 0.63 & 2.75 \\
\hline
\end{tabular}
\caption{$\chi^2$ results in all channels using reweighted MC}
\label{tab:chi2Rew}
\end{table}



%%%\subsubsection{Conclusion}

To establish the compatibility between the unfolded $p_{\mathrm{T}}^{Z}$  measurements using dilepton and hadronic recoil observables, covariance matrices between the p$_{\mathrm{T}}^{Z}(ll)$ and p$_\mathrm{T}^{Z}(u_T)$ measurements are built using their statistical and systematic uncertainties. The covariance matrices are dominated by the statistical uncertainty and unfolding bias uncertainty. Two important features can be observed: large bin-to-bin migration is seen in the statistical uncertainty because of the finite resolution of the measurements, and the dilepton and hadronic recoil methods are highly correlated because of conservation of transverse momentum at Born level.

We see that $\chi^2$/dof is consistent with 1 at 5 TeV in both channels and at 13 TeV in the $Z\rightarrow ee$ channel. The 13 TeV $Z\rightarrow \mu\mu$ channel is worst agreed between the p$_{\mathrm{T}}^{Z}(ll)$ and p$_\mathrm{T}^{Z}(u_T)$ .% The unfolding bias uncertainty is still being finalized which should lead to an improved $\chi^2$/dof in this 13 TeV $Z\rightarrow \mu\mu$ channel.

%\subsubsection{Compatibility study of the $p_{T}^{Z}$ measurements using $Z\rightarrow ee$ and $Z\rightarrow \mu\mu$ channel}
\label{sssec:compatibilityEEMuMu}
	
	{\color{blue}  This paragraph to move in the combination session ??? }
	
The measurements of $p_{T}^{Z}$ distribution are processed in two leptonic decayed channels:  $Z\rightarrow ee$ and $Z\rightarrow \mu\mu$.
The compatibility of the measurements in two channels are important validations for combination of the $p_{T}^{Z}$ results.
The measured distributions $p_{T}^{ll}$ in two channels are shown in \Fig{\ref{f:pTeeMuMu}}.
This section is based on 2GeV binning results.

\begin{figure}[h]
  \centering
  \includegraphics[width=.4\textwidth]{figure/Zeemumu_Compatibility/pTEEMuMu_5TeV_acc.pdf}
  \includegraphics[width=.4\textwidth]{figure/Zeemumu_Compatibility/pTEEMuMu_13TeV_acc.pdf}
  \caption{
  Compatibility of the $p_{T}^{Z}$ measurement in $Z\rightarrow ee$ and $Z \rightarrow \mu\mu$ channels. The left plot is for 5TeV and the right plot is for 13TeV.}
    \label{f:pTeeMuMu}
\end{figure}

Given that $Z\rightarrow ee$ and $Z\rightarrow \mu\mu$ channels are complete uncorrelated, covariance matrices are estimated independently in each channel and summed afterwards. 
Only statistic uncertainties are taken into account in the compatibility, shown in \Fig{\ref{f:CovMatricesEEMuMu}}.

\begin{figure}[h]
  \centering
  \includegraphics[width=.4\textwidth]{figure/Zeemumu_Compatibility/covMatrix_stat_5TeV_stat_acc.pdf}
  \includegraphics[width=.4\textwidth]{figure/Zeemumu_Compatibility/covMatrix_stat_13TeV_stat_acc.pdf}
  \caption{Statistic covariance matrices of the $p_{T}^{Z}$ measurement in $Z\rightarrow ee$ and $Z\rightarrow \mu\mu$ channels. The left plot is for 5TeV and the right plot is for 13TeV.}
    \label{f:CovMatricesEEMuMu}
\end{figure}


\begin{table}[h] 
 \centering
\begin{tabular}{|c|c|c|}
\hline
  & 5TeV & 13TeV\\
 \hline
 $\chi^2$ & 13.2 & 20.5 \\
 \hline
$\chi^2/$ dof & 0.55 & 0.85\\
\hline
\end{tabular}
\caption{$\chi^2$ results of $p_{T}^{Z}$ measurements in $Z\rightarrow ee$ and $Z\rightarrow \mu\mu$ channels.}
\label{tab:chi2highmu}
\end{table}

%%%%%%%%% \subsubsection{Conclusion}
The $\chi^2/dof$ is 0.633 at 5~TeV and 0.975 at 13~TeV. $p_{T}^{ll}$measurements in $Z\rightarrow ee$ and $Z\rightarrow \mu\mu$ channels are well agreed.
