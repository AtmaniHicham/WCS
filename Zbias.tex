%%%%%%%%%%%%%%%%%%%%%%%%%%%%%%%%%%%%%
\label{ss:ZpT_bias_rew}
%%%%%%%%%%%%%%%%%%%%%%%%%%%%%%%%%%%%%
%%%%%%%%%%%%%%%%%%%%%%%%%%%%%%%%%%%%%


The unfolding bias uncertainty on the unfolded \ptz spectrum is estimated following the data driven procedure described in Section~\ref{sec:zbias}.
%%%%%%%%%%%%%%%%%%%%%%%%%%%%%%%%%%%%%
\subsubsection{Unfolding bias uncertainty at $\sqrt{s} = 13$~\TeV }
%%%%%%%%%%%%%%%%%%%%%%%%%%%%%%%%%%%%%
After the baseline \POWHEG signal MC is corrected with the data-driven correction described in Section~\ref{sec:zbias} (see also section~\ref{sec:physcorrWpTrew}), the reweighting procedure is repeated to estimate the remaining bias (that should be negligible by construction), as well as its related uncertainty.
Figure~\ref{fig:fits_13_1GeV} shows the central fit function and its associated variations used to estimate the corresponding fit uncertainties.

\begin{figure}[h]
\centering
\subfloat[\Zee.]{\includegraphics[width=.45\textwidth]{figure/ZpT_bias_rew/nonClosureBias/RewFactorWithVariations_13TeV_Zee_v20210511_BiasCorrOn_checknonclosure_finerT.pdf}}
\subfloat[\Zmm.]{\includegraphics[width=.45\textwidth]{figure/ZpT_bias_rew/nonClosureBias/RewFactorWithVariations_13TeV_Zmumu_v20210511_BiasCorrOn_checknonclosure_finerT.pdf}}\\
\caption{Fitted reweighting function with its independent variations obtained from propagating the fit uncertainties to the four parameters at $\sqrt{s} = 13$~\TeV, taking into account the correlation between the fit parameters. The black curve corresponds to the central fit, and the colored curves correspond to the variations of fit function. }
\label{fig:fits_13_1GeV}
\end{figure}

The estimated bias correction and bias uncertainty are shown in Figure~\ref{fig:biasunc_13}. The bias uncertainty is calculated as the quadratic sum of the results from varied fit functions.
For the \ptdilep measurement, an uncertainty of $<0.1\%$ is found in the first bin after two iterations. For the \ut measurement (with $5$~\GeV\ bin width), an uncertainty of $\sim 0.3\% -0.4\%$  is found after five unfolding iterations.
The bias correction is estimated using the central fit function and is negligible by construction.

\begin{figure}[h]
\centering
\subfloat[\Zee, \ptll.]{\includegraphics[width=.4\textwidth]{figure/ZpT_bias_rew/nonClosureBias/Bias_var_ZpT_Unfolding_v20210511_13TeV_Zee_2GeVBin_2pTiters_5uTiters_pTll.pdf}}
\subfloat[\Zmm, \ptll.]{\includegraphics[width=.4\textwidth]{figure/ZpT_bias_rew/nonClosureBias/Bias_var_ZpT_Unfolding_v20210511_13TeV_Zmumu_2GeVBin_2pTiters_5uTiters_pTll.pdf}}\\
\subfloat[\Zee, \ut.]{\includegraphics[width=.4\textwidth]{figure/ZpT_bias_rew/nonClosureBias/Bias_var_ZpT_Unfolding_v20210511_13TeV_Zee_5GeVBin_2pTiters_5uTiters_uT.pdf}}
\subfloat[\Zmm, \ut.]{\includegraphics[width=.4\textwidth]{figure/ZpT_bias_rew/nonClosureBias/Bias_var_ZpT_Unfolding_v20210511_13TeV_Zmumu_5GeVBin_2pTiters_5uTiters_uT.pdf}}
\caption{Bias correction and uncertainty (relative to the bias correction) on \ptz at $\sqrt{s} = 13$~\TeV. The results are shown after two unfolding iteration for the dilepton \pt\ measurement and five unfolding iterations for the \ut measurement. }
\label{fig:biasunc_13}
\end{figure}

In addition to the uncertainty from the variations of the fit function (fit parameter uncertainty), the following sources of uncertainties are tested (as in the \Wboson\ case):
\begin{itemize}
\item the variation on the fit parametrisation, based on the testing of alternative fit functions (other than the baseline fit). Figure~\ref{fig:fit_nominalMC_diffFunc_13_1GeV} shows the set of all alternative fit functions that are tried.
\item the variation over the initial (\ptz, $y$) distribution, based on the testing of different MC predictions to which the baseline truth-level (\ptz, $y$) distribution is reweighted in 2D. Figure~\ref{fig:fit_nominalMC_diffMC_13_1GeV} shows the fitted reweighting functions for different MC predictions: \textsc{Pythia}, \textsc{Sherpa}, \textsc{Herwig7}, DyTurbo using the CT10 and NNPDF3.0 PDFs.
\end{itemize}

\begin{figure}[h]
\centering
\subfloat[\Zee.]{\includegraphics[width=.45\textwidth]{figure/ZpT_bias_rew/NonRewMC/alternativeBiasUnc/RewFunc_MCGen_13TeV_Zee_nonrewMC_v20210511_v3.pdf}}
\subfloat[\Zmm.]{\includegraphics[width=.45\textwidth]{figure/ZpT_bias_rew/NonRewMC/alternativeBiasUnc/RewFunc_MCGen_13TeV_Zmumu_nonrewMC_v20210511_v3.pdf}}\\
\caption{Fitted reweighting functions, $w_{T}$, using different MC predictions obtained after $\chi^2$ minimisation at $\sqrt{s} = 13$~\TeV.}
\label{fig:fit_nominalMC_diffMC_13_1GeV}
\end{figure}

The reweighted versions of \POWHEG\ to different MC spectra display large differences with respect to the data at the reconstructed level. Therefore, different reweighting fit functions are used during the $\chi^{2}$ minimisation to get the best data-to-MC agreement in each case.
Equation~\ref{ZeqRwfuncGP} is used as the best fit function for the reweighting of the predictions from \textsc{Pythia} and \textsc{DyTurbo} using the CT10 PDF set. A double gaussian added to a second-order polynomial is used for the spectra predicted by \textsc{Herwig7} and \textsc{DyTurbo} using the NNPDF3.0 PDF set, see equation~\ref{eq:Wdblgaus}.
The reweighted baseline MC to the \Sherpa\ spectrum is discarded because it has unphysical jumps.
The agreement between data and MC for all considered uncertainty variations after the data-driven \pt-reweighting using the best fitting function for each of them is shown in Figure~\ref{fig:fit_datamc_altunc_13_1GeV}.

\begin{figure}[h]
\centering
\subfloat[\Zee, \ptll.]{\includegraphics[width=.45\textwidth]{figure/ZpT_bias_rew/NonRewMC/alternativeBiasUnc/DataMC_ZpT_Unfolding_v20210511_13TeV_Zee_2GeVBin_2pTiters_2uTiters_FullSetBiasv3Envelop_pTll.pdf}}
\subfloat[\Zmm, \ptll.]{\includegraphics[width=.45\textwidth]{figure/ZpT_bias_rew/NonRewMC/alternativeBiasUnc/DataMC_ZpT_Unfolding_v20210511_13TeV_Zmumu_2GeVBin_2pTiters_2uTiters_FullSetBiasv3Envelop_pTll.pdf}}\\
\subfloat[\Zee, \ut.]{\includegraphics[width=.45\textwidth]{figure/ZpT_bias_rew/NonRewMC/alternativeBiasUnc/DataMC_ZpT_Unfolding_v20210511_13TeV_Zee_5GeVBin_10pTiters_10uTiters_FullSetBiasv3Envelop_uT.pdf}}
\subfloat[\Zmm, \ut.]{\includegraphics[width=.45\textwidth]{figure/ZpT_bias_rew/NonRewMC/alternativeBiasUnc/DataMC_ZpT_Unfolding_v20210511_13TeV_Zmumu_5GeVBin_10pTiters_10uTiters_FullSetBiasv3Envelop_uT.pdf}}\\
\caption{Ratio of data to MC simulation at reconstructed level before (black dashed line) and after applying the data-driven \pt\ reweighting (colored line) for \ptll\ and \ut\ at $\sqrt{s} = 13$~\TeV, for all uncertainty variations. The dashed colored lines represent the functions that are not considered in the final bias uncertainty. }
\label{fig:fit_datamc_altunc_13_1GeV}
\end{figure}

The bias uncertainty from each of these variations is calculated as a difference between the unfolded result with respect to the result using the baseline MC and fit function.
These are shown using an optimized number of unfolding iterations in Figure~\ref{fig:bias_altunc_13}.
An envelope, maximum from all considered contributions in each bin, is taken for the final bias uncertainty and added in a quadrature to the baseline fit parameters uncertainty, as in the \Wboson\ case.

%\begin{figure}[h]
%\centering
%\subfloat[]{\includegraphics[width=.45\textwidth]{figure/ZpT_bias_rew/NonRewMC/alternativeBiasUnc/Bias_diffFits_ZpT_Unfolding_v20210511_13TeV_Zee_2GeVBin_2pTiters_2uTiters_FullSetBiasv3Envelop_pTll.pdf}}
%\subfloat[]{\includegraphics[width=.45\textwidth]{figure/ZpT_bias_rew/NonRewMC/alternativeBiasUnc/Bias_diffFits_ZpT_Unfolding_v20210511_13TeV_Zmumu_2GeVBin_2pTiters_2uTiters_FullSetBiasv3Envelop_pTll.pdf}}\\
%\caption{Bias uncertainty from the alternative sources on \ptz using \ptdilep measurement for the \Zee and \Zmm channels at $\sqrt{s} = 13$~\TeV. The results are shown after two unfolding iterations. Dashed line represent the uncertainties that are not included to the final uncertainty.}
%\label{fig:bias_altunc_ptLL_13}
%\end{figure}

\begin{figure}[h]
\centering
\subfloat[\Zee, \ptll.]{\includegraphics[width=.45\textwidth]{figure/ZpT_bias_rew/NonRewMC/alternativeBiasUnc/Bias_diffFits_ZpT_Unfolding_v20210511_13TeV_Zee_2GeVBin_2pTiters_2uTiters_FullSetBiasv3Envelop_pTll.pdf}}
\subfloat[\Zmm, \ptll.]{\includegraphics[width=.45\textwidth]{figure/ZpT_bias_rew/NonRewMC/alternativeBiasUnc/Bias_diffFits_ZpT_Unfolding_v20210511_13TeV_Zmumu_2GeVBin_2pTiters_2uTiters_FullSetBiasv3Envelop_pTll.pdf}}\\
\subfloat[\Zee, \ut, 5~\GeV\ binning.]{\includegraphics[width=.45\textwidth]{figure/ZpT_bias_rew/NonRewMC/alternativeBiasUnc/Bias_diffFits_ZpT_Unfolding_v20210511_13TeV_Zee_5GeVBin_15pTiters_15uTiters_FullSetBiasv3Envelop_uT.pdf}}
\subfloat[\Zmm, \ut, 5~\GeV\ binning.]{\includegraphics[width=.45\textwidth]{figure/ZpT_bias_rew/NonRewMC/alternativeBiasUnc/Bias_diffFits_ZpT_Unfolding_v20210511_13TeV_Zmumu_5GeVBin_15pTiters_15uTiters_FullSetBiasv3Envelop_uT.pdf}}\\
\subfloat[\Zee, \ut, 7~\GeV\ binning.]{\includegraphics[width=.45\textwidth]{figure/ZpT_bias_rew/NonRewMC/alternativeBiasUnc/Bias_diffFits_ZpT_Unfolding_v20210511_13TeV_Zee_finerT_Rebin7_10pTiters_10uTiters_FullSetBiasv3Envelop_uT.pdf}}
\subfloat[\Zmm, \ut, 7~\GeV\ binning.]{\includegraphics[width=.45\textwidth]{figure/ZpT_bias_rew/NonRewMC/alternativeBiasUnc/Bias_diffFits_ZpT_Unfolding_v20210511_13TeV_Zmumu_finerT_Rebin7_10pTiters_10uTiters_FullSetBiasv3Envelop_uT.pdf}}\\
%\subfloat[]{\includegraphics[width=.45\textwidth]{figure/ZpT_bias_rew/NonRewMC/alternativeBiasUnc/Bias_diffFits_ZpT_Unfolding_v20210511_13TeV_Zee_finerT_Rebin8_10pTiters_10uTiters_FullSetBiasv3Envelop_uT.pdf}}
%\subfloat[]{\includegraphics[width=.45\textwidth]{figure/ZpT_bias_rew/NonRewMC/alternativeBiasUnc/Bias_diffFits_ZpT_Unfolding_v20210511_13TeV_Zmumu_finerT_Rebin8_10pTiters_10uTiters_FullSetBiasv3Envelop_uT.pdf}}\\
\caption{Bias uncertainty from the variations on the parametrisation and initial (\pt, y) spectrum on \ptz measurements at $\sqrt{s} = 13$~\TeV. The results are shown after two unfolding iterations for the \ptdilep\ measurement. The \ut\ measurement uses fifteen unfolding iterations for the $5$~\GeV\ binnning and ten unfolding iterations for the $7$~\GeV\ binning. The dashed lines represent the uncertainties that are not included in the final uncertainty.}
\label{fig:bias_altunc_13}
\end{figure}

The number of iterations is optimised to minimise the quadratic sum of the bias and of the statistical uncertainties.
The statistical, bias and total (statistical and bias) uncertainties as a function of the number of iterations in the first four bins are shown in Figure~\ref{fig:biasunc_min_ptll_13} and~\ref{fig:biasunc_min_uT5GeV_13} for the \ptdilep and \ut measurements, respectively.
The optimization of the number of unfolding iterations for the \ut measurement using wider bins are given in Appendix~\ref{sec:zpt_bias_new}.
Two unfolding iterations are selected for the \ptdilep measurement and fifteen ($5$~\GeV\ binning) or ten ($7$~\GeV\ and $8$~\GeV\ binnings) unfolding iterations are selected for the \ut measurement.
In general, the bias uncertainty is $<1\%$ for both measurements, with the main contribution arising from the \textsc{Herwig7} or \textsc{DyTurbo NNPDF3.0} variations.


\begin{figure}[h]
\centering
\subfloat[\Zee.]{\includegraphics[width=.5\textwidth]{figure/ZpT_bias_rew/BiasVsIter/BiasUnc_Vs_Iter_13TeV_Zee_2GeVBin_pT_BinNumber_4.pdf}}
\subfloat[\Zmm.]{\includegraphics[width=.5\textwidth]{figure/ZpT_bias_rew/BiasVsIter/BiasUnc_Vs_Iter_13TeV_Zmumu_2GeVBin_pT_BinNumber_4.pdf}}\\
\caption{Bias (red line), statistical (blue line) and total (black line) uncertainties as a function of the number of unfolding iterations for the \ptdilep measurement at $\sqrt{s} = 13$~\TeV\ . The uncertainties are shown for the first four bins.}
\label{fig:biasunc_min_ptll_13}
\end{figure}

\begin{figure}[h]
\centering
\subfloat[\Zee.]{\includegraphics[width=.5\textwidth]{figure/ZpT_bias_rew/BiasVsIter/BiasUnc_Vs_Iter_13TeV_Zee_5GeVBin_uT_BinNumber_4.pdf}}
\subfloat[\Zmm.]{\includegraphics[width=.5\textwidth]{figure/ZpT_bias_rew/BiasVsIter/BiasUnc_Vs_Iter_13TeV_Zmumu_5GeVBin_uT_BinNumber_4.pdf}}\\
\caption{Bias (red line), statistical (blue line) and total (black line) uncertainties as a function of the number of unfolding iterations for the \ut measurement at $\sqrt{s} = 13$~\TeV\ using the $5$~\GeV\ binning configuration. The uncertainties are shown for the first four bins.}
\label{fig:biasunc_min_uT5GeV_13}
\end{figure}

%%%%%%%%%%%%%%%%%%%%%%%%%%%%%%%%%%%%%%%%%
%%%%%%%%%%%%%%%%%%%%%%%%%%%%%%%%%%%%%%%%%
\subsubsection{Unfolding bias uncertainty at $\sqrt{s} = 5$~\TeV}
The unfolding bias uncertainty at $\sqrt{s} = 5$~\TeV\ is estimated in the same way as at $\sqrt{s} = 13$~\TeV.

The central and varied fit functions are shown in Figure~\ref{fig:fits_5_1GeV}.
The bias correction and bias uncertainties are shown after two unfolding iterations for \ptz using the \ptdilep\ and the \ut\ measurements in Figure~\ref{fig:biasunc_5}.
An uncertainty of $\sim 0.01\% -0.02\%$ and of $\sim 0.2\% $ is found in the first bin after two iterations for the \ptdilep and \ut (with $5$~\GeV\ bin width) measurements, respectively.
The bias correction is estimated using the central fit function and is negligible, by construction.

\begin{figure}[h]
\centering
\subfloat[\Zee.]{\includegraphics[width=.45\textwidth]{figure/ZpT_bias_rew/nonClosureBias/RewFactorWithVariations_5TeV_Zee_v20210511_BiasCorrOn_checknonclosure_finerT.pdf}}
\subfloat[\Zmm.]{\includegraphics[width=.45\textwidth]{figure/ZpT_bias_rew/nonClosureBias/RewFactorWithVariations_5TeV_Zmumu_v20210511_BiasCorrOn_checknonclosure_finerT.pdf}}
\caption{Fitted reweighting function with its independent variations obtained from propagating the fit uncertainties to the two parameters at $\sqrt{s} = 5$~\TeV, taking into account the correlation between the fit parameters. The black curve corresponds to the central fit, and the colored curves correspond to the variations of fit function.}
\label{fig:fits_5_1GeV}
\end{figure}

\begin{figure}[h]
\centering
\subfloat[\Zee, \ptll.]{\includegraphics[width=.4\textwidth]{figure/ZpT_bias_rew/nonClosureBias/Bias_var_ZpT_Unfolding_v20210511_5TeV_Zee_2GeVBin_2pTiters_2uTiters_pTll.pdf}}
\subfloat[\Zmm, \ptll.]{\includegraphics[width=.4\textwidth]{figure/ZpT_bias_rew/nonClosureBias/Bias_var_ZpT_Unfolding_v20210511_5TeV_Zmumu_2GeVBin_2pTiters_2uTiters_pTll.pdf}}\\
\subfloat[\Zee, \ut.]{\includegraphics[width=.4\textwidth]{figure/ZpT_bias_rew/nonClosureBias/Bias_var_ZpT_Unfolding_v20210511_5TeV_Zee_5GeVBin_2pTiters_2uTiters_uT.pdf}}
\subfloat[\Zmm, \ut.]{\includegraphics[width=.4\textwidth]{figure/ZpT_bias_rew/nonClosureBias/Bias_var_ZpT_Unfolding_v20210511_5TeV_Zmumu_5GeVBin_2pTiters_2uTiters_uT.pdf}}
\caption{Bias correction and uncertainty (relative to the bias correction) on \ptz at $\sqrt{s} = 5$~\TeV. The results are shown after two unfolding iteration for the \ptdilep and  \ut measurements. }
\label{fig:biasunc_5}
\end{figure}

Similarly as at $\sqrt{s} = 13$~\TeV,  the following sources of uncertainties are tested (and similarly as in the \Wboson\ case, still):
\begin{itemize}
\item the variation on the fit parametrisation, based on the testing of alternative fit functions (other than the baseline fit). Figure~\ref{fig:fit_nominalMC_diffFunc_13_1GeV} shows the set of all alternative fit functions that are tried.
\item the variation over the initial (\ptz, $y$) distribution, based on the testing of different MC predictions to which the baseline truth-level (\ptz, $y$) distribution is reweighted in 2D. Figure~\ref{fig:fit_nominalMC_diffMC_5_1GeV} shows the fitted reweighting functions for different MC predictions: \textsc{Pythia}, \textsc{Sherpa}, \textsc{Herwig7}, DyTurbo using the CT10 and NNPDF3.0 PDFs.
\end{itemize}

Equation~\ref{ZeqRwfuncP} is used as the best fit function for the reweighting starting from the spectrum reweighted to \textsc{Pythia}.  Equation~\ref{ZeqRwfuncGP} is used for the spectrum reweighted to DyTurbo using the CT10 PDF set.
A double gaussian added to a second-order polynomial is used for the \textsc{Herwig7} and DyTurbo NNPDF3.0 predictions, see equation~\ref{eq:Wdblgaus}.
\Sherpa\ is again discarded because of the presence of unphysical jumps in its \pttruthv\ spectrum.
The DyTurbo prediction using the CT10 PDF set is discarded as well, because of a poor data-to-MC agreement after performing the reweighting.
The agreement between data and MC for all considered variations after applying the reweighting using the best fitting function is shown in Figure~\ref{fig:fit_datamc_altunc_5_1GeV}.

\begin{figure}[h]
\centering
\subfloat[\Zee.]{\includegraphics[width=.45\textwidth]{figure/ZpT_bias_rew/NonRewMC/alternativeBiasUnc/RewFunc_MCGen_5TeV_Zee_nonrewMC_v20210511_v2.pdf}}
\subfloat[\Zmm.]{\includegraphics[width=.45\textwidth]{figure/ZpT_bias_rew/NonRewMC/alternativeBiasUnc/RewFunc_MCGen_5TeV_Zmumu_nonrewMC_v20210511_v2.pdf}}
\caption{Fitted reweighting functions, $w_{T}$, using different MC predictions obtained after $\chi^2$ minimisation at $\sqrt{s} = 5$~\TeV. }
\label{fig:fit_nominalMC_diffMC_5_1GeV}
\end{figure}

\begin{figure}[h]
\centering
\subfloat[\Zee, \ptll.]{\includegraphics[width=.45\textwidth]{figure/ZpT_bias_rew/NonRewMC/alternativeBiasUnc/DataMC_ZpT_Unfolding_v20210511_5TeV_Zee_2GeVBin_2pTiters_2uTiters_FullSetBiasv3Envelop_pTll.pdf}}
\subfloat[\Zmm, \ptll.]{\includegraphics[width=.45\textwidth]{figure/ZpT_bias_rew/NonRewMC/alternativeBiasUnc/DataMC_ZpT_Unfolding_v20210511_5TeV_Zmumu_2GeVBin_2pTiters_2uTiters_FullSetBiasv3Envelop_pTll.pdf}}\\
\subfloat[\Zee, \ut.]{\includegraphics[width=.45\textwidth]{figure/ZpT_bias_rew/NonRewMC/alternativeBiasUnc/DataMC_ZpT_Unfolding_v20210511_5TeV_Zee_5GeVBin_5pTiters_5uTiters_FullSetBiasv3Envelop_uT.pdf}}
\subfloat[\Zmm, \ut.]{\includegraphics[width=.45\textwidth]{figure/ZpT_bias_rew/NonRewMC/alternativeBiasUnc/DataMC_ZpT_Unfolding_v20210511_5TeV_Zmumu_5GeVBin_5pTiters_5uTiters_FullSetBiasv3Envelop_uT.pdf}}\\
\caption{Ratio of data to MC simulation at reconstructed level before (black dashed line) and after applying the data-driven \pt\ reweighting (colored line) for \ptll\ and \ut\ at $\sqrt{s} = 5$~\TeV, for all uncertainty variations. The dashed colored lines represent the functions that are not considered in the final bias uncertainty.}
\label{fig:fit_datamc_altunc_5_1GeV}
\end{figure}


The bias uncertainty from each of these variations using an optimized number of unfolding iterations are shown in Figure~\ref{fig:bias_altunc_5}.
An envelope, maximum from all considered contributions in each bin, is taken for the final bias uncertainty and added in a quadrature to the baseline fit parameters uncertainty, as in the \Wboson\ case.

%\begin{figure}[h]
%\centering
%\subfloat[]{\includegraphics[width=.45\textwidth]{figure/ZpT_bias_rew/NonRewMC/alternativeBiasUnc/Bias_diffFits_ZpT_Unfolding_v20210511_5TeV_Zee_2GeVBin_2pTiters_2uTiters_FullSetBiasv3Envelop_pTll.pdf}}
%\subfloat[]{\includegraphics[width=.45\textwidth]{figure/ZpT_bias_rew/NonRewMC/alternativeBiasUnc/Bias_diffFits_ZpT_Unfolding_v20210511_5TeV_Zmumu_2GeVBin_2pTiters_2uTiters_FullSetBiasv3Envelop_pTll.pdf}}\\
%\caption{Bias uncertainty from the alternative sources on \ptz using \ptdilep measurement for the \Zee and \Zmm channels at $\sqrt{s} = 5$~\TeV. The results are shown after two unfolding iterations. Dashed line represent the uncertainties that are not included to the final uncertainty.}
%\label{fig:bias_altunc_ptLL_5}
%\end{figure}

\begin{figure}[h]
\centering
\subfloat[\Zee, \ptll.]{\includegraphics[width=.45\textwidth]{figure/ZpT_bias_rew/NonRewMC/alternativeBiasUnc/Bias_diffFits_ZpT_Unfolding_v20210511_5TeV_Zee_2GeVBin_2pTiters_2uTiters_FullSetBiasv3Envelop_pTll.pdf}}
\subfloat[\Zmm, \ptll.]{\includegraphics[width=.45\textwidth]{figure/ZpT_bias_rew/NonRewMC/alternativeBiasUnc/Bias_diffFits_ZpT_Unfolding_v20210511_5TeV_Zmumu_2GeVBin_2pTiters_2uTiters_FullSetBiasv3Envelop_pTll.pdf}}\\
\subfloat[\Zee, \ut, 5~\GeV\ binning.]{\includegraphics[width=.45\textwidth]{figure/ZpT_bias_rew/NonRewMC/alternativeBiasUnc/Bias_diffFits_ZpT_Unfolding_v20210511_5TeV_Zee_5GeVBin_5pTiters_5uTiters_FullSetBiasv3Envelop_uT.pdf}}
\subfloat[\Zmm, \ut, 5~\GeV\ binning.]{\includegraphics[width=.45\textwidth]{figure/ZpT_bias_rew/NonRewMC/alternativeBiasUnc/Bias_diffFits_ZpT_Unfolding_v20210511_5TeV_Zmumu_5GeVBin_5pTiters_5uTiters_FullSetBiasv3Envelop_uT.pdf}}\\
\subfloat[\Zee, \ut, 7~\GeV\ binning.]{\includegraphics[width=.45\textwidth]{figure/ZpT_bias_rew/NonRewMC/alternativeBiasUnc/Bias_diffFits_ZpT_Unfolding_v20210511_5TeV_Zee_finerT_Rebin7_5pTiters_5uTiters_FullSetBiasv3Envelop_uT.pdf}}
\subfloat[\Zmm, \ut, 7~\GeV\ binning.]{\includegraphics[width=.45\textwidth]{figure/ZpT_bias_rew/NonRewMC/alternativeBiasUnc/Bias_diffFits_ZpT_Unfolding_v20210511_5TeV_Zmumu_finerT_Rebin7_5pTiters_5uTiters_FullSetBiasv3Envelop_uT.pdf}}\\
%\subfloat[]{\includegraphics[width=.45\textwidth]{figure/ZpT_bias_rew/NonRewMC/alternativeBiasUnc/Bias_diffFits_ZpT_Unfolding_v20210511_5TeV_Zee_finerT_Rebin8_5pTiters_5uTiters_FullSetBiasv3Envelop_uT.pdf}}
%\subfloat[]{\includegraphics[width=.45\textwidth]{figure/ZpT_bias_rew/NonRewMC/alternativeBiasUnc/Bias_diffFits_ZpT_Unfolding_v20210511_5TeV_Zmumu_finerT_Rebin8_5pTiters_5uTiters_FullSetBiasv3Envelop_uT.pdf}}\\
\caption{Bias uncertainty from the variations on the parametrisation and initial (\pt, y) spectrum on \ptz measurements at $\sqrt{s} = 5$~\TeV. The results are shown after two unfolding iterations for the \ptdilep\ measurement. The \ut\ measurement uses five unfolding iterations for both the $5$~\GeV\ and $7$~\GeV\ binnings. The dashed lines represent the uncertainties that are not included in the final uncertainty.}
\label{fig:bias_altunc_5}
\end{figure}

The optimization of the number of unfolding iterations is shown in Figure~\ref{fig:biasunc_min_ptll_5} and~\ref{fig:biasunc_min_uT5GeV_5} for the \ptdilep and \ut measurements, respectively.
The optimization of the number of unfolding iterations for the \ut measurement using wider bins are given in Appendix~\ref{sec:zpt_bias_new}.
Two unfolding iterations are selected for the \ptdilep\ measurement and five unfolding iterations are selected for the \ut\ measurement.
In general, the bias uncertainty is $\sim 1\%$ in the \ptdilep\ measurement and in the muon channel for \ut\  measurement, while it is larger in the electron channel for the \ut\ measurement.
The main contribution comes from the variations using \textsc{Herwig7} or \textsc{DyTurbo NNPDF3.0}.

\begin{figure}[h]
\centering
\subfloat[\Zee.]{\includegraphics[width=.5\textwidth]{figure/ZpT_bias_rew/BiasVsIter/BiasUnc_Vs_Iter_5TeV_Zee_2GeVBin_pT_BinNumber_4.pdf}}
\subfloat[\Zmm.]{\includegraphics[width=.5\textwidth]{figure/ZpT_bias_rew/BiasVsIter/BiasUnc_Vs_Iter_5TeV_Zmumu_2GeVBin_pT_BinNumber_4.pdf}}\\
\caption{Bias (red line), statistical (blue line) and total (black line) uncertainties as the function of the unfolding iteration for the \ptdilep measurement at $\sqrt{s} = 5$~\TeV. The uncertainties are shown for the first four bins.}
\label{fig:biasunc_min_ptll_5}
\end{figure}

\begin{figure}[h]
\centering
\subfloat[\Zee.]{\includegraphics[width=.5\textwidth]{figure/ZpT_bias_rew/BiasVsIter/BiasUnc_Vs_Iter_5TeV_Zee_5GeVBin_uT_BinNumber_4.pdf}}
\subfloat[\Zmm.]{\includegraphics[width=.5\textwidth]{figure/ZpT_bias_rew/BiasVsIter/BiasUnc_Vs_Iter_5TeV_Zmumu_5GeVBin_uT_BinNumber_4.pdf}}\\
\caption{Bias (red line), statistical (blue line) and total (black line) uncertainties as the function of the unfolding iteration for \ut measurement at $\sqrt{s} = 5$~\TeV\ using $5$~\GeV\ binning configuration. The uncertainties are shown for the first four bins.}
\label{fig:biasunc_min_uT5GeV_5}
\end{figure}
