\section{\ptz Non-reweighted unfolded results comparison }
\label{sec:zpt_nonrew_unf}

Here we present the differential unfolded cross-section results comparing the three different unfolded datasets:
\begin{enumerate}
  \item the unfolded data using the \pt-corrected MC, see section~\ref{sec:physcorrWpTrew},
  \item the unfolded data using the non-corrected MC,
  \item the data that has been unfolded using the non-corrected MC, and only after that unfolding, corrected for the unfolding bias.
\end{enumerate}

 The bias-reweighting should only reduce the uncertainty coming from the unfolding bias, and should not meaningfully affect the data itself within the bias uncertainty. The following figures, beginning with figure \ref{f:biascompare_Zee_pT13}, show these total and normalized unfolded differential cross section for both \Zee and \Zmm at both 13 and 5 TeV for each observable. As can be seen in the ratio panels, these checks confirm that each of the three unfolded datasets remain consistent within uncertainty.

\begin{figure}[h]
\centering
\subfloat[]{\includegraphics[width=.49\textwidth]{figure/ZpT_app_nonrew/BiasCompare_pTll_Zee_13TeV_2GeVbins_2iters.pdf}\label{f:biascompare_Zee_pT13}}
\subfloat[]{\includegraphics[width=.49\textwidth]{figure/ZpT_app_nonrew/BiasCompare_normalized_pTll_Zee_13TeV_2GeVbins_2iters.pdf}\label{f:biascompare_norm_Zee_pT13}}

\subfloat[]{\includegraphics[width=.49\textwidth]{figure/ZpT_app_nonrew/BiasCompare_pTll_Zmumu_13TeV_2GeVbins_2iters.pdf}\label{f:biascompare_Zmm_pT13}}
\subfloat[]{\includegraphics[width=.49\textwidth]{figure/ZpT_app_nonrew/BiasCompare_normalized_pTll_Zmumu_13TeV_2GeVbins_2iters.pdf}\label{f:biascompare_norm_Zmm_pT13}}
\caption{Unfolded differential cross-sections using \ptdilep for the three cases described in the text : using the \pt-reweighted MC (black, used in the final result), using the non \pt-corrected results (blue), and accounting for the bias after the unfolding (green). The reweighted (purple) and non-reweighted (red) MCs are also both shown. \ptdilep is unfolded using Bayesian unfolding with two unfolding iterations. The denominator in the ratio panel is the unfolded data using the \pt-reweighted signal MC, so the ratio is Data or MC to Data. The uncertainty band is the total uncertainty (without the luminosity uncertainty). Results are shown for both \Zee (top) and \Zmm (bottom) for both total (left) and normalized (right) unfolded cross-sections at 13 TeV.}\end{figure}

\begin{figure}[h]
\centering
\subfloat[]{\includegraphics[width=.49\textwidth]{figure/ZpT_app_nonrew/BiasCompare_uT_Zee_13TeV_5GeVbins_20iters.pdf}\label{f:biascompare_Zee_uT13}}
\subfloat[]{\includegraphics[width=.49\textwidth]{figure/ZpT_app_nonrew/BiasCompare_normalized_uT_Zee_13TeV_5GeVbins_20iters.pdf}\label{f:biascompare_norm_Zee_uT13}}

\subfloat[]{\includegraphics[width=.49\textwidth]{figure/ZpT_app_nonrew/BiasCompare_uT_Zmumu_13TeV_5GeVbins_20iters.pdf}\label{f:biascompare_Zmm_uT13}}
\subfloat[]{\includegraphics[width=.49\textwidth]{figure/ZpT_app_nonrew/BiasCompare_normalized_uT_Zmumu_13TeV_5GeVbins_20iters.pdf}\label{f:biascompare_norm_Zmm_uT13}}
\caption{Unfolded differential cross-sections using \ut for the three cases described in the text : using the \pt-reweighted MC (black, used in the final result), using the non \pt-corrected results (blue), and accounting for the bias after the unfolding (green). The reweighted (purple) and non-reweighted (red) MCs are also both shown. \ut is unfolded using Bayesian unfolding with 20 unfolding iterations. The denominator in the ratio panel is the unfolded data using the \pt-reweighted signal MC, so the ratio is Data or MC to Data. The uncertainty band is the total uncertainty (without the luminosity uncertainty). Results are shown for both \Zee (top) and \Zmm (bottom) for both total (left) and normalized (right) unfolded cross-sections at 13 TeV.}\end{figure}

\begin{figure}[h]
\centering
\subfloat[]{\includegraphics[width=.49\textwidth]{figure/ZpT_app_nonrew/BiasCompare_pTll_Zee_5TeV_2GeVbins_2iters.pdf}\label{f:biascompare_Zee_pT5}}
\subfloat[]{\includegraphics[width=.49\textwidth]{figure/ZpT_app_nonrew/BiasCompare_normalized_pTll_Zee_5TeV_2GeVbins_2iters.pdf}\label{f:biascompare_norm_Zee_pT5}}

\subfloat[]{\includegraphics[width=.49\textwidth]{figure/ZpT_app_nonrew/BiasCompare_pTll_Zmumu_5TeV_2GeVbins_2iters.pdf}\label{f:biascompare_Zmm_pT5}}
\subfloat[]{\includegraphics[width=.49\textwidth]{figure/ZpT_app_nonrew/BiasCompare_normalized_pTll_Zmumu_5TeV_2GeVbins_2iters.pdf}\label{f:biascompare_norm_Zmm_pT5}}
\caption{Unfolded differential cross-sections using \ptdilep for the three cases described in the text : using the \pt-reweighted MC (black, used in the final result), using the non \pt-corrected results (blue), and accounting for the bias after the unfolding (green). The reweighted (purple) and non-reweighted (red) MCs are also both shown. \ptdilep is unfolded using Bayesian unfolding with two unfolding iterations. The denominator in the ratio panel is the unfolded data using the \pt-reweighted signal MC, so the ratio is Data or MC to Data. The uncertainty band is the total uncertainty (without the luminosity uncertainty). Results are shown for both \Zee (top) and \Zmm (bottom) for both total (left) and normalized (right) unfolded cross-sections at 5 TeV.}\end{figure}

\begin{figure}[h]
\centering
\subfloat[]{\includegraphics[width=.49\textwidth]{figure/ZpT_app_nonrew/BiasCompare_uT_Zee_5TeV_5GeVbins_2iters.pdf}\label{f:biascompare_Zee_uT5}}
\subfloat[]{\includegraphics[width=.49\textwidth]{figure/ZpT_app_nonrew/BiasCompare_normalized_uT_Zee_5TeV_5GeVbins_2iters.pdf}\label{f:biascompare_norm_Zee_uT5}}

\subfloat[]{\includegraphics[width=.49\textwidth]{figure/ZpT_app_nonrew/BiasCompare_uT_Zmumu_5TeV_5GeVbins_2iters.pdf}\label{f:biascompare_Zmm_uT5}}
\subfloat[]{\includegraphics[width=.49\textwidth]{figure/ZpT_app_nonrew/BiasCompare_normalized_uT_Zmumu_5TeV_5GeVbins_2iters.pdf}\label{f:biascompare_norm_Zmm_uT5}}
\caption{Unfolded differential cross-sections using \ut for the three cases described in the text : using the \pt-reweighted MC (black, used in the final result), using the non \pt-corrected results (blue), and accounting for the bias after the unfolding (green). The reweighted (purple) and non-reweighted (red) MCs are also both shown. \ut is unfolded using Bayesian unfolding with 20 unfolding iterations. The denominator in the ratio panel is the unfolded data using the \pt-reweighted signal MC, so the ratio is Data or MC to Data. The uncertainty band is the total uncertainty (without the luminosity uncertainty). Results are shown for both \Zee (top) and \Zmm (bottom) for both total (left) and normalized (right) unfolded cross-sections at 5 TeV.}\end{figure}
