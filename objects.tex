
\section{Objects definitions}
\label{sec:objects}

In this analysis, electrons or muons are considered as leptons. Electrons are reconstructed from energy clusters in the calorimeter associated to a track, while muons are reconstructed as a track that crosses the muon chambers. Both objects are required to have $\pT>25~\GeV$. Additionally, electrons should have $|\eta|<2.47$ (excluding the crack region:$1.37<|\eta|<1.52$). Muons should fulfills $|\eta|<2.4$.
The electrons candidates must satisfy the Medium LH identification criteria and are also required to be isolated, satisfying $\texttt{ptvarcone20}<0.1\times \pT$ track isolation criteria.
%Electrons are also required to satisfy the $\texttt{ptvarcone20}<0.1\times \pT$ track isolation criteria.
%\item [-] Electrons are also required to satisfy the FCLoose isolation criteria : the track isolation must satisfy $\texttt{ptcone20\_TightTTVA\_pt1000}<0.15\times \pT$; the calorimeter isolation must satisfy $\texttt{topoetcone20}<0.20\times \pT$, using a cone of $\Delta R=0.20$.
 The Muons candidates  must satisfy the Medium identification criteria and are also required to be isolated, satisfying $\texttt{ptvarcone20}<0.1\times \pT$ track isolation criteria.
%\item [-] Muons are also required to satisfy the FCLoose\_FixedRad isolation criteria : the track isolation must satisfy $\texttt{ptcone30\_TightTTVA\_pt1000}<0.15\times \pT$ for muons with $\pT<50~\GeV$, and $\texttt{ptcone20\_TightTTVA\_pt1000}<0.15\times \pT$ otherwise; the calorimeter isolation must satisfy $\texttt{topoetcone20}<0.30\times \pT$, using a cone of $\Delta R=0.20$.

Leptons are required to originate from the primary vertex. The longitudinal impact parameter of each lepton track, defined as the distance between the track and the primary vertex along the beam line multiplied by the sine of the track $\theta$ angle, is required to be less than $0.5$ mm ( $\Delta z_{0}\times \sin(\theta) <0.5$). Furthermore, the significance of the
transverse impact parameter, defined by the transverse impact parameter ($d_{0}$ ) of a lepton track with respect
to the beam line, divided by its estimated uncertainty ($\sigma(d_{0})$ ), is required to satisfy for electrons  $\sigma(d_{0}) < 5$ and for muons  $\sigma(d_{0}) < 3$.

Dedicated lepton calibrations and efficiency corrections are applied to the data and the MC used in the analysis following the Reference~\cite{Xu:2657152,Sydorenko:2657116} respectively for electrons and muons. A short summary of the calibration strategy and the corresponding resulting uncertainties for the electrons and muons is described in Section~\ref{sec:elCorr} and ~\ref{sec:muCorr} respectively.

In the context of W boson production the need of a high precision computation of the neutrino kinematics is mandatory, and this is done with the help of the \textit{hadronic recoil}. In proton-proton collisions there is a non-zero transverse momentum for vector boson production which is originated by initial (gluon/quark) state radiation in the transverse plane, \textit{i.e.}, described by the relation:
\begin{equation}
\vec{p}_{T}(W/Z)=\vec{p}_{T}^{\mathrm{lepton1}}+\vec{p}_{T}^{\mathrm{lepton2}}=-\sum \vec{p}_{T}^{\mathrm{ISR quark,gluons}} = -\vec{\ut},
\end{equation}

where $p_{T}(W/Z)$ denotes the transverse momentum of the W or Z boson and $p_{T}^{\mathrm{lepton}}$ denote the transverse momenta decay leptons. The  \textit{hadronic recoil}, $\sum \vec{p}_{T}^{\mathrm{ISR quark,gluons}}$,  is the quantity which accounts for all transverse momenta of the partons from initial state radiation, and is denoted as $\vec{\ut}$.
The  \textit{hadronic recoil} has been used successfully for the W mass measurement, where it was proven to be a powerful tool to determine the neutrino transverse momentum in a more precise way w.r.t. the TST based algorithm~\cite{Aaboud:2017svj}. An improvement brought in this analysis, as compared to this RunI measurement, is the use of \emph{particle flow objects} (PFOs) as input constituents to the building of \ut, which used to be recontructed from the vector sum of all topo-clusters in~\cite{Aaboud:2017svj}.

It is possible to determine indirectly the neutrino $\nu$ transverse momentum using the following expression:

\begin{equation}\label{eq:METandNeutrino}
\vec{E}^{miss}_{T}:=\vec{p}_{T}^{\nu}=-(\vec{u}+p_{T} ^{l^{\pm}})
\end{equation}

where $\vec{E}^{miss}_{T}$ is the missing transverse energy. The transverse momenta of additional muons with $\pT>5~\GeV$ and satisfying the medium identification is added at the definition of \MET{} while their calorimeter energy deposit is subtracted from the event.

A critical quantity is \set, that represents the total event
activity and is related to the resolution of the \ut\
measurement. \set\ is defined as the scalar sum of the \pt of all
PFOs. Naturally \set\ will be strongly dependent on the vector boson
dynamics and grow with $\ptv=\ut$. To disentangle these effects, an
even more important quantity is defined by $\setue = \set -
\ut$. \setue\ may be thought of the event activity corrected for the
``directed'' recoil activity and thus represents the activity from
underlying event, pileup, and emissions beyond first (hard)
emission.~\footnote{E.g. a $V+1$ jet event has $\setue \sim 0$ on
  parton level for any jet \pt and thus a finite $\setue > 0$ is due to additional
  activity.}


The detailed description of the \textit{hadronic recoil} definition and calibration is done in reference~\cite{Li:2657182} and also summarised briefly in Section~\ref{sec:recoilCorr}.

\subsection{Association of the signal leptons to the primary vertex}\label{sec:pvcorr}
The association of the signal leptons to the primary vertex is performed through impact parameters requirements, as indicated above.
The probability to reconstruct a primary vertex associated with the hard interaction in \Wboson\ boson events will depend on
the hadronic activity produced in addition to the \Wboson\ boson, $i.e.$ the hadronic recoil and the underlying event. In this respect, the single lepton from the
\Wboson\ boson decay is not sufficient to reconstruct a primary vertex. In addition, the track associated to the lepton might not
fulfill the track requirements in the vertex finding. This can be especially the case for electrons. In this case, the reconstructed
vertex might not be selected as the event primary vertex over vertices due to pileup.
By contrast, in \Zboson\ boson selected events, largely used to calibrate the \Wboson\ analyses under study, the presence of two reconstructed leptons ensures that there is almost no selection inefficiency due to the effects mentioned before.
Thus a purely \Zboson-based calibration will not correct for this effect. Furthermore,
the efficiency of this requirement of a primary vertex with an associated single lepton in
\Wboson\ events is not well reproduced in our simulation.
This can be for example inferred from figure~\ref{fig:n_Z}, that shows a different fraction of events with 0 and 1 additional tracks matched to the $z$-position of the hard interaction in the data as compared to the MC simulations in \Zmm\ events.
\begin{figure}[hb]
\begin{center}
\subfloat[Fraction of events with 0 additional track.]{\label{fig:n_Z_b}\includegraphics[width=0.42\textwidth]{figure/pvlep/frag0_vs_pt.pdf}}\hspace*{10pt}
\subfloat[Fraction of events with 1 additional track.]{\label{fig:n_Z_c}\includegraphics[width=0.42\textwidth]{figure/pvlep/frag1_vs_pt.pdf}}
\caption{Fraction of additional tracks matched to the hard interaction in $z$ as a function of \ptmm\ in \Zmm\ events at 13~\TeV.\label{fig:n_Z}}
\end{center}
\end{figure}

Therefore, a dedicated correction of this effect is required.
The full details of this correction are explained in~\cite{Kretzschmar:2657141}.
It relies on the extraction of the efficiency to match the lepton to the primary vertex as a function of boson \pt, by using three categories of events : with 0, 1 or $>1$ additional matched track to the $z$-position of the hard interaction.
The inefficiency in each category is extracted in \Wboson\ simulation, while the fraction of events belonging to each category can be extracted in \Zboson\ events, separately for the data and for the simulation.

One has to note that the mismodeling of this efficiency can be largely explained by a difference in the modeling of underlying event, and that the
uncertainty on this correction is conservatively taken into account by looking at the difference in the results when using either \SHERPA\ or \POWPYTHIA (generator systematic uncertainty). One can refer to~\cite{Kretzschmar:2657141} for more details on this.


\subsection{Electrons calibration and correction}\label{sec:elCorr}

Reference~\cite{Xu:2657152} details the electrons performance studies done using the special low-pile-up datasets collected by the ATLAS detector at
$\sqrt{s} = 5$ and 13~\TeV.
The electron reconstruction scale factors are obtained by extrapolation of the standard high-pileup SFs to the \lowmu regime, and apply to both the 13 and 5 \TeV{} datasets. The identification scale factors are measured in-situ separately for the 13 and 5 \TeV{} data.
Isolation and trigger efficiencies and SF are measured in-situ for the 13 and 5 \TeV{} data sets.
For the electron energy calibration the global strategy follows the methodology applied in high-$\mu$ standard calibration and described in Ref.~\cite{Aad:2019tso}.
Electron energy scale and resolution corrections are measured in-situ using \Zee events from the low pile-up dataset. 
%%The main limitation of the insitu performance studies is the limited statistic of the  \Zee samples in each of the 5 and 13~\TeV datasets.


\subsection{Muons calibration and correction}\label{sec:muCorr}


Reference~\cite{Sydorenko:2657116} details the muon-related performance measurements obtained using the ATLAS special with low pile-up conditions at $\sqrt{s} = 5$
TeV and $13$ TeV in view of this measurement.
Muon reconstruction and \textit{track-to-vertex association} requirements efficiencies measured
insitu with \Zmm\ events are found to be compatible with the
corresponding high-$\mu$ measurements and are measured to
permille-level precision. The muon trigger and isolation efficiencies
are measured insitu in the \lowmu\ data, for each dataset, to typically better than $<1\%$ precision, limited
by the size of the \Zmm\ data samples. Finally, the muon calibration
derived from the high-$\mu$ data has been cross-checked using both
\Zmm\ and \Jmumu\ events in the \lowmu\ datasets and has been found to be adequate at the
current level of systematics. A dedicated correction for
charge-dependent momentum biases, so-called sagitta correction, has
been derived using a combination of different methods using the $13$ TeV \lowmu dataset.


\subsection{Hadronic Recoil calibration and correction}\label{sec:recoilCorr}
The detailed procedure to calibrate the hadronic recoil is described in~\cite{Li:2657182}. It is briefly summarised here. The calibration is obtained as a function of \setue\ and \ptv\ of the boson, since the hadronic recoil is mainly sensitive to these. Since the correction is obtained in \Zboson\ events, this dependence also allows to extrapolate to the \Wboson\ events, that have different \setue\ and \ptv\ distributions.
%

The procedure consists of three steps :
\begin{itemize}
	\item First, the \setue\ distribution in the Monte-Carlo should be well modelled and match that of the data. More precisely, it is crucial to model correctly the correlation between \setue\ and \ptv, since we want to have a good description of the activity as a function of our measured physics observable. This is achieved thanks to a 2-dimensional reweighting, obtained in \Zboson\ events. In the simulated \Wboson\ events, an additional reweighting of \setue\ in bins of \ut\ is applied. A further 1-dimensional reweighting is obtained for each process (\Wminus, \Wplus\ and \Zboson) to recover the initial underlying \pttruth\ spectrum.

	\item Second, the direction of the recoil is corrected, taking the projections on $x$ and $y$ axes of the recoil in \Zboson\ events, and correcting for any data to MC differences.

	\item Finally, response and resolution corrections are, once again, obtained in-situ in \Zboson\ events, where the parallel and perpendicular components can be extracted in the data as a function of \setue\ and \ptll, and compared to the Monte-Carlo to extract corrective coefficients.
\end{itemize}

This \Zboson\ boson based calibration is applied to \Wboson\ events ; uncertainties due to this extrapolation are included. A summary on these uncertainties, as well as their impact on the unfolded spectra, are discussed in section~\ref{subsec:uncsummary}.

